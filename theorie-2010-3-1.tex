\ifx \mpreamble \undefined
\documentclass[12pt,a4paper]{article}
\usepackage{answers}
%\usepackage{microtype}
\usepackage[left=3cm,top=2cm,bottom=3cm,right=2cm,includehead,includefoot]{geometry}

\usepackage{amsfonts,amsmath,amssymb,amsthm,graphicx}
\usepackage[utf8]{inputenc}
\usepackage{ngerman}

\usepackage{pstricks}
\usepackage{pst-circ}
\usepackage{pst-plot}

\ifx \envfinal \empty
\usepackage{pst-pdf}
\fi

\usepackage{booktabs}

% Muss als letztes eingebunden werden
%\usepackage[bookmarks=true,bookmarksnumbered,colorlinks=true,pdftitle={IPhO-Aufgaben},pdfstartview=FitH,pdfauthor={Pavel Zorin}]{hyperref}
\usepackage[bookmarks=false,pdftitle={IPhO-Aufgabensammlung},pdfstartview=FitH,pdfauthor={Pavel Zorin}]{hyperref}

%Times 10^n
\newcommand{\ee}[1]{\cdot 10^{#1}}
%Units
\newcommand{\unit}[1]{\,\mathrm{#1}}
%Differential d's
\newcommand{\dif}{\mathrm{d}}
\newcommand{\tdif}[2]{\frac{\dif#1}{\dif#2}}
\newcommand{\pdif}[2]{\frac{\partial#1}{\partial#2}}
\newcommand{\ppdif}[2]{\frac{\partial^{2}#1}{\partial#2^{2}}}
%Degree
\newcommand{\degr}{^\circ}
%Degree Celsius (C) symbol
\newcommand{\cel}{\,^\circ\mathrm{C}}
% Hinweis
\newcommand{\hinweis}{\emph{Hinweis:} }
% Aufgaben mit Buchstaben numerieren
\newenvironment{abcenum}{\renewcommand{\labelenumi}{(\alph{enumi})} \begin{enumerate}}{\end{enumerate}\renewcommand{\labelenumi}{\theenumi .}}
%%%%%%%%%% Skizzen %%%%%%%%%%%%
%\ifx \envfinal \empty
%%% Final
%\else
%%% Vorschau
%\fi

\def \mpreamble {}
\else
%\ref{test}
\fi
\ifx \envfinal \undefined


\newcommand{\skizze}[1]{
\begin{figure}
\begin{center}
#1
\end{center}
\end{figure}
}




%\documentclass[12pt,a4paper]{article}
\newcounter{numlabel}
\setcounter{numlabel}{0}

\newcommand{\problemlabel}{}
\newenvironment{problem}[2]{
\stepcounter{numlabel}
\renewcommand{\problemlabel}{Aufgabe \the\value{numlabel}: #1}
\subsubsection*{\problemlabel \emph{(#2 Punkte)}}
}{}
\newenvironment{solution}{\subsubsection*{\problemlabel}}{}
\newenvironment{expsolution}{\subsubsection*{\problemlabel}}{}

\begin{document}

\fi

\begin{problem}{Gasvolumina}{2,5}
Ein ideales Gas durchläuft den abgebildeten Prozess.

%Prozess

Bestimmen Sie das Verhältnis von maximalem zu minimalem Volumen, das das Gas einnimmt.
 \begin{solution}
  
 \end{solution}
\end{problem}

\begin{problem}{Kugel am Faden}{5}
Auf einem horizontalen Tisch ruht reibungsfrei ein Brett mit einem Stativ, an dem ein Faden der Länge $l$ mit einer kleinen Kugel am Ende hängt.  Das Brett mit Stativ hat die gleiche Masse wie die Kugel.
% TODO Bild
Bestimmen Sie die minimale Geschwindigkeit, die man der Kugel geben muss, damit sie sich mit gespanntem Faden einmal vollständig um die Befestigung am Stativ drehen kann.
\end{problem}

\begin{problem}{Spezifischer Widerstand}{5,5}
Eine Firma stellt Bleche eines Materials mit einer sehr geringen Dicke $d$ her.  Um den spezifischen Widerstand des Materials zu bestimmen, setzt man vier Metallnadeln in gleichen Abständen $l$, wie in der Abbildung, auf ein Blech und verbindet die Nadeln $1$ und $4$ mit einer Stromquelle, die eine Stromst rke $I$ zur Verfügung stellt.  Darüber hinaus wird die Spannung zwischen den Nadeln $2$ und $3$ gemessen.

Leiten Sie einen Ausdruck für den spezifischen Widerstand des Materials als Funktion der gegebenen bzw. gemessen Größen ab.  Die Abbildung zeigt nur einen Ausschnitt des Bleches.  Sie können annehmen, dass das Blech insgesamt sehr groß ist und dass der Kontaktwiderstand zwischen dem Blech und den Nadeln vernachlässigbar ist.
\end{problem}

\begin{problem}{Solarofen}{5,5}
Ein parabolischer Spiegel dessen Durchmesser der vierfachen Brennweite von $f=1\unit{m}$ entspricht fokussiert nahezu senkrecht einfallendes Sonnenlicht in einen kleinen Raumbereich um den Brennpunkt.

\begin{abcenum}
  \item Der Winkeldurchmesser der Sonne beträgt von der Erde aus gesehen etwa $0,5^{\circ}$.  Schätzen Sie den Durchmesser $d$ des Raumbereiches ab, in den das Sonnenlicht hinein fokussiert wird.
  \itme In den Fokus des Spiegels wird ein kleiner, schwarzer, kugelförmiger und wärmeleitfähiger Körper vom Durchmesser $d$ platziert.  Schätzen Sie ab, auf welchen Anteil der Sonnentemperatur dieser Körper maximal aufgeheizt werden kann.
\end{abcenum}
\end{problem}

\begin{problem}{Lautstaerke von Lautsprechern}{4,5}
  Eine Firma hatte sich auf punktfömrige Lautsprecher für empfindliche Ohren spezialisiert, die Schallwellen isotrop in alle Richtungen abgeben und in einem Meter Entfernung einen Schalldruckpegel von $1\unit{dB}$ besitzen. Der Schlalldruckpegel kann als Maß für die Lautstärke der Lautsprecher verwendet werden.
  \begin{abcenum}
    \item Ein Entwickler hat zwei dieser Lautsprecher kohärent direkt nebeneinander gestellt und versucht zu klären, wie groß die Lautstärke der beiden Lautsprecher gemeinsam ist.

Bestimmen Sie den maximal erreichbaren gemeinsamen Schalldruckpegel der zwei Lautsprecher in einer Entfernung von einem Meter.

\item Die Nachfrage nach Lautsprechern, die höhere Lautstärken abgeben, ist doch höher als vermutet und die Firma möchte aus ihren Lautsprechern Systeme herstellen, die auch größere Lautstärken bedienen.

Geben Sie an, wie viele der punktförmigen Lautsprecher man kohärent direkt nebeneinander stellen muss, um in einer Entfernung von einem Meter einen Schalldruckpegel von mindestens $n\unit{dB}$ zu erreichen. Berechnen Sie außerdem, wie viele Lautsprecher nötig sind, um so einen Schalldruckpegel von $80\unit{dB}$ zu erzeugen und geben Sie Ihre Einschätzung zu der Frage ab, ob die Firma mit ihrer Idee Erfolg haben wird.

\item Ein Kunde hat ein System aus zwei Lautsprechern gekauft und findet es nun doch zu laut.  Er bestellt den Kundenservice und wünscht, dass sein Lautsprechersystem auf $1\unit{dB}$ heruntergetrimmt wird.  Da die Lautsprecher fest verkabelt sind, kann man nicht einfach einen der Lautsprecher ausschalten, sondern lediglich mit einer Trimmschraube den Phasenwinkel der Schallwellen einstellen.

Bestimmen Sie, auf welchen Phasenwinkel die Trimmschraube eingestellt werden muss, um die gewünschte Lautstärke zu erreichen!

Alternativ könnte man die Lautsprecher auch weiter weg aufstellen.  Geben Sie an, in welcher Entfernung von dem Lautsprechersystem aus zwei Lautsprechern (ohne verstelltem Phasenwinkel) der Schalldruckpegel $1 \unit{dB}$ beträgt.
  \end{abcenum}

\hinweis Ein deziBel ist ein Zehntel Bel.  Der Bel-Wert einer Schallwelle ist der Zehnerlogarithmus des Quadrats des Quotienten der Schalldruckamplitude der Welle und einer festen Schalldruckamplitude $\hat{p}_0$, die die Wahrnemungsgrenze des menschlichen Gehörs angibt, also:
\begin{equation*}
  \mbox{Schalldruckpegel in Bel\ } = \lg\left(\frac{\hat{p}^2}{\hat{p}_0^2}\right)
\end{equation*}

Die Schalldrücke zweier kohärenter Wellen addieren sich beim Überlagern.
\end{problem}

\begin{problem}{Beruhigung einer Waescheschleuder}{7}
Waschmaschinen und Wäscheschleuder können aufgrund der in der Regel nicht gleichmäßigen Beladung hüpfen und wandern. Dieser Effekt kann z.B. durch elastische, gedämpfte Lagerung der Trommel reduziert werden.

Die nebenstehende Abbildung zeigt ein ebenes Modell einer Wäscheschleuder, das im Folgenden betrachtet werden soll.

%TODO: Bild

Dabei bezeichnet $O$ das gehäusefeste Drehlager.  Der Trommelmittelpunkt $M$ kann gegenüber $O$ elastische Querbewegungen ausführen.  $S$ ist der Schwerpunkt der befüllten Trommel, die eine Masse von $m_t=10\unit{kg}$ besitzt.  Der Abstand $MS$ betrage stets $1\unit{cm}$.  Die Federn in $x$- und $y$-Richtung weisen die gleiche Steifigkeit von $k=25\unit{kNm^{-1}}$ auf.  Dämpfungskräfte und die Erdanziehungskraft sollen im Folgenden vernachlässigt werden.  Die maximale Drehzahl beträgt $1500\unit{\frac U{min}}$.

\begin{abcenum}
  \item Bestimmen Sie die Eigenfrequenzen der Trommelquerschwingungen in $x$- und $y$-Richtung im Stand.  Diese entsprechen den kritischen Drehzahlen des Systems.
\item Leiten Sie einen Ausdruck für die radiale Auslenkung $r$ des Trommelmittelpuntks $M$ gegenüber $O$ in Abhängigkeit von der Drehzahl $n$ ab.
\item Geben Sie den Wert für $r$ für die Drezahlen $n=150\unit{\frac{U}{min}}; 1500\unit{\frac{U}{min}}$ an und zeichnen Sie für beide Fälle in die unten stehende nicht maßstabsgetreue Abbildung ein, wie die Punkte $O, M, S$ zueinander liegen.  Verwenden sie zur Kennzeichnung des Punktes $O$ ein Kreuz.
\item Geben Sie an, welche Auslenkung bei sehr hohen Drehzahlen erreicht wird.
\end{abcenum}

Im Folgenden soll eine weitere Möglichkeit zum Verringern der Wucht des Systems betrachtet werden.  Als Wucht wird dabei das Produkt aus dem Abstand des Trommelmittelpunktes zu dem Schwerpunkt der beladenen Trommel und der Masse der beladenen Trommel bezeichnet.  Nehmen Sie dazu an, dass sich zusätzlich zwei gleiche Kugeln jeweils der Masse $m=0,25\unit{kg}$ reibungsfrei auf einer kreisförmigen Bahn mit Radius $R=25\unit{cm}$ um den Mittelpunkt $M$ der Trommel bewegen können (siehe Abbildung oben). Die Bahn ist mit der Trommel fest verbunden und soll so gestaltet sein, dass die Kugeln nicht kollidieren, sondern nebeneinander vorbeilaufen können.

\begin{abcenum}
  \item Skizzieren Sie die möglichen Gleichgewichtszustände dieses Systems bei der Drehung der Trommel und notieren Sie, bei welchen ein Auswuchten stattfindet.  Wie stehen bei den möglichen Gleichgewichtszuständen jeweils $O, M, S$ sowie die beiden Kugeln zueinander?
  \item Geben Sie an, welche Gleichgewichtszustände sich bei einer kleinen Drehzahl von $n=150\unit{\frac{U}{min}}$ sowie bei einer großen Drehzahl $n=1500\unit{\frac{U}{min}}$ einstellen.
  \item Stellen Sie eine Beziehung auf, die die zusätzlichen Massen erfüllen müssen, um ein vollständiges Auswuchten zu ermölgichen.  Begründen Sie außerdem, ob das Auswuchten auf diese Weise für alle Drehzahlen möglich ist.
\end{abcenum}

\end{problem}

\ifx \envfinal \undefined
\end{document}
\fi

