\ifx \mpreamble \undefined
\documentclass[12pt,a4paper]{article}
\usepackage{answers}
\usepackage{microtype}
\usepackage[left=3cm,top=2cm,bottom=3cm,right=2cm,includehead,includefoot]{geometry}

\usepackage{amsfonts,amsmath,amssymb,amsthm,graphicx}
\usepackage[utf8]{inputenc}
\usepackage[T1]{fontenc}
\usepackage{ngerman}

\usepackage{pstricks}
\usepackage{pst-circ}
\usepackage{pst-plot}
%\usepackage{pst-node}
\usepackage{booktabs}

%Times 10^n
\newcommand{\ee}[1]{\cdot 10^{#1}}
%Units
\newcommand{\unit}[1]{\,\mathrm{#1}}
%Differential d's
\newcommand{\dif}{\mathrm{d}}
\newcommand{\tdif}[2]{\frac{\dif#1}{\dif#2}}
\newcommand{\pdif}[2]{\frac{\partial#1}{\partial#2}}
\newcommand{\ppdif}[2]{\frac{\partial^{2}#1}{\partial#2^{2}}}
%Degree
\newcommand{\degr}{^\circ}
%Degree Celsius (C) symbol
\newcommand{\cel}{\unit{^\circ C}}
% Hinweis
\newcommand{\hinweis}{\emph{Hinweis:} }
% Aufgaben mit Buchstaben numerieren
\newenvironment{abcenum}{\renewcommand{\labelenumi}{(\alph{enumi})} \begin{enumerate}}{\end{enumerate}\renewcommand{\labelenumi}{\theenumi .}}

\title{IPhO-Aufgabensammlung}
\author{Zusammengestellt von Pavel Zorin\\
unter Verwendung der Aufzeichungen von\\
Bastian Hacker, Igor Gotlibovych, Tobias Holder,\\
Patrick Steinmüller,\\
sowie anonymer Quellen}

\ifx \envfinal \undefined % Preview mode
\newcommand{\skizze}[1]{
\begin{figure}
\begin{center}
#1
\end{center}
\end{figure}
}

\renewcommand{\thesection}{}
\renewcommand{\thesubsection}{\arabic{subsection}}

\newenvironment{problem}[2]{
\subsection{#1 \emph{(#2 Punkte)}}
}{}
\newenvironment{solution}{\subsection*{Lösung}}{}
\newenvironment{expsolution}{\subsection*{Lösung}}{}

\else % Final mode

\usepackage{pst-pdf}

\usepackage{hyperref}
\hypersetup{
bookmarks=true,
pdfpagemode=UseNone,
pdfstartview=FitH,
pdfdisplaydoctitle=true,
pdflang=de-DE,
pdfborder={0 0 0}, % No link borders
unicode=true,
pdftitle={IPhO-Aufgabensammlung},
pdfauthor={Pavel Zorin},
pdfsubject={Aufgaben der 3. und 4. Runden der deutschen Auswahl zur IPhO},
pdfkeywords={}
}

\newcommand{\skizze}[1]{
\begin{center}
#1
\end{center}
}

%%%%%%%%%%%%%%%%%%%%%%%%%%%%%%%%%%%%%%%%%%%%%%%%%%%%%%%%%%%%%%%%%%%%%%%%%%%%%%%%%%%%%%%%%%%
%%%%%%%%%%%%%%%%%%%%%%%%%%%%%%%%%%%%%%%%%%%%%%%%%%%%%%%%%%%%%%%%%%%%%%%%%%%%%%%%%%%%%%%%%%%
%
%     Formatierung der Aufgaben/Lösungen
%
%     Anmerkung dazu: nicht-ASCII-Zeichen in Überschriften gehen aus rätselhaften Gründen
%     nicht. Sie werden zwar bei der Aufgabe richtig angezeigt, in die Lösungsdatei wird
%     aber eine unverständliche Sequenz geschrieben, die dann nicht wieder gelesen werden
%     kann.
%
%%%%%%%%%%%%%%%%%%%%%%%%%%%%%%%%%%%%%%%%%%%%%%%%%%%%%%%%%%%%%%%%%%%%%%%%%%%%%%%%%%%%%%%%%%%
%%%%%%%%%%%%%%%%%%%%%%%%%%%%%%%%%%%%%%%%%%%%%%%%%%%%%%%%%%%%%%%%%%%%%%%%%%%%%%%%%%%%%%%%%%%
\renewcommand{\thesubsection}{\arabic{subsection}}
\def\solTitle<#1><#2>{
\renewcommand{\thesubsection}{#1}
\subsection{#2}
\renewcommand{\thesubsection}{\arabic{subsection}}
}

\newenvironment{problem}[2]{
\subsection{#1 \emph{(#2 Punkte)}}
\renewcommand{\Currentlabel}{<\thesubsection><#1>}
}{}
\Newassociation{solution}{Soln}{solutions}
\Newassociation{expsolution}{Soln}{expsolutions}
\renewenvironment{Soln}[1]{
\solTitle #1
}{}

\fi

\def \mpreamble {}
\begin{document}

\else
%\ref{test}
\fi


%\subsection*{2007 -- 4. Runde -- Theoretische Klausur II (12.04.2007)}
\begin{problem}{Fadenpendel}{2,5}
\skizze{
\psset{unit=0.3cm}
\begin{pspicture}(-3,-10.2)(11,0.2)
\psline[linewidth=0.1](0,0)(8.66,-5)
\psarc[linestyle=dashed]{<-}(0,0){10}{270}{330}
\psarc[linestyle=dashed]{<-}(0,-7){3}{60}{270}
\psdots[dotsize=0.3cm](8.66,-5)
\psdots[dotsize=0.12cm](0,-7)(0,0)
\rput(4.9,-2){$R$}
\psline[linestyle=dashed]{->}(0,0)(0,-7)
\uput[r](0,-3){$r$}
\psline{<->}(9.6,-5)(9.6,-10)
\uput[r](9.6,-7.5){$h$}
\end{pspicture}
}
Ein Fadenpendel der Länge $R$ wird in der Höhe $h$ losgelassen und soll eine Kreisbewegung um einen im Abstand $r=0.7 R$ senkrecht unter dem Aufhängepunkt angebrachten Nagel ausführen. Unter welchen Bedingungen an $h$ ist solche Bewegung möglich? Man vernachlässige die Effekte der Aufwicklung des Fadens auf den Nagel.
\begin{solution}
Energieerhaltung:
\[
mgh = mg\cdot 2(R-r)+\frac12 mv^2
\]
Zentripedalbeschleunigung:
\[
\frac{v^2}{R-r} = g
\]
\[
h = \frac 5 2 (R-r) = \frac 3 4 R
\]
\end{solution}
\end{problem}

\begin{problem}{Kreisprozess}{4,5}
\skizze{
\psset{unit=0.35cm}
\begin{pspicture}(-1.5,-1)(13,11)
\psline{->}(0,0)(0,10)
\psline{->}(0,0)(12,0)
\psline(3.5,2)(10.5,2)
\psline[linestyle=dashed](0,2)(3.5,2)
\psline(1,9)(3,9)
\psline[linestyle=dashed](0,9)(1,9)
\psplot{1}{3.5}{x -1.2 exp 9 mul}
\psplot{3}{10.5}{x 3 div -1.2 exp 9 mul}
\rput(12,-0.7){$V$}
\rput(-0.7,10){$p$}
\rput(-0.7,2){$p_1$}
\rput(-0.7,9){$p_2$}
\end{pspicture}
}
Der in der Abbildung angegebener Kreisprozess besteht aus zwei isobaren und zwei adiabatischen Zustandsänderungen, die im Uhrzeigersinn abgelaufen werden. Das Arbeitsmedium ist dabei ein ideales Gas. Geben Sie den Wirkungsgrad des Prozesses in Abhängigkeit von $p_1$, $p_2$ und $c_p$ an.
%\begin{solution}
% ?
%\end{solution}
\end{problem}

\begin{problem}{DA-Wandler}{4}
\skizze{
\psset{unit=0.7cm}
\begin{pspicture}(0,2)(7,10)
\pnode(1,3){U3}
\pnode(1,5){U2}
\pnode(1,7){U1}
\pnode(5,3){Erde1}
\pnode(3,9){Erde2}
\pnode(3,3){M3}
\pnode(3,5){M2}
\pnode(3,7){M1}
\pnode(5,7){Uaus}
%Widerstände bei den Spannungsquellen
\resistor(U3)(M3){$2 R$}
\resistor(U2)(M2){$2 R$}
\resistor(U1)(M1){$2 R$}
%Widerstände in der Mitte von unten nach oben
\resistor[labeloffset=0.7](Erde1)(M3){$2 R$}
\resistor[labeloffset=-0.7](M3)(M2){$R$}
\resistor[labeloffset=-0.7](M2)(M1){$R$}
\resistor[labeloffset=-0.7](M1)(Erde2){$2 R$}

\ground(Erde1)
\ground{180}(Erde2)

\pscircle*(U1){1mm}
\pscircle*(U2){1mm}
\pscircle*(U3){1mm}

\rput(0.5,3){$\varphi_3$}
\rput(0.5,5){$\varphi_2$}
\rput(0.5,7){$\varphi_1$}


\pscircle*(Uaus){1mm}
\uput[r](Uaus){$\varphi_{\mathrm{aus}}$}

\wire(M1)(Uaus)
\end{pspicture}
}
Der oberste und unterste Punkt der Schaltung ist jeweils geerdet. An den drei Eingängen links kann das Potenzial $0\unit{V}$ oder $1\unit{V}$ angelegt werden, wobei $0\unit{V}$ Erdung bedeutet. Man berechne das Ausgangspotenzial für alle möglichen Kombinationen der Eingangsspannungen.

\begin{solution}
Man bezeichne das Potential in der vertikalen Widerstandskette rechts von $\varphi_2$ mit $\varphi_5$ und rechts von $\varphi_3$ mit $\varphi_6$. Dreimalige Anwendung der Knotenregel ($\sum\frac U R = 0$) liefert:
\[
 \frac 1 2 (0-\varphi_{\mathrm{aus}}) + \frac 1 2 (\varphi_1-\varphi_{\mathrm{aus}}) + (\varphi_5-\varphi_{\mathrm{aus}}) = 0
\]
\[
 (\varphi_{\mathrm{aus}}-\varphi_5) + \frac 1 2 (\varphi_2-\varphi_5) + (\varphi_6-\varphi_5) = 0
\]
\[
 (\varphi_5-\varphi_6) + \frac 1 2 (\varphi_3-\varphi_6) + \frac 1 2 (0-\varphi_6) = 0
\]

\[
 \varphi_{\mathrm{aus}} = \frac 1 3 \left(\varphi_1+\frac 1 2 \varphi_2+\frac 1 4 \varphi_3\right)
\]
\end{solution}
\end{problem}


\begin{problem}{Rotierende Zylinder}{4}
Zwei gleich lange Zylinder bekannter Radien und Massen rotieren um parallele Achsen mit bekannten Winkelgeschwindigkeiten. Nun wird eine der Rotationsachsen solange verschoben, bis sich die zwei Zylinder berühren. Welche Winkelgeschwindigkeiten stellen sich dabei ein? Was passiert im Spezialfall gleicher Zylinder und gleicher Anfangswinkelgeschwindigkeiten?
\begin{solution}
% ?
Trägheitsmomente:
\[
 J_1 = \frac 1 2 m_1\,r_1^2 \qquad\qquad J_2 = \frac 1 2 m_2\,r_2^2
\]
Die gleiche Reibungskraft $F$ wirkt auf beide Zylinder mit dem Drehmoment $M = F\cdot r$ und somit dem Drehmomentstoß $\Delta L = M\cdot\Delta t = J\cdot\Delta\omega$:
\[
 \frac{J_1\cdot(\omega_1'-\omega_1)}{r_1} = \frac{J_2\cdot(\omega_2'-\omega_2)}{r_2}
\]
Am Ende ist die Differenzgeschwindigkeit 0:
\[
 \omega_1'\cdot r_1 + \omega_2'\cdot r_2 = 0
\]
\[
 \omega_1' = \frac{m_1r_1\omega_1-m_2r_2\omega_2}{m_1r_1+m_2r_1}
\]
Bei gleichen Ausgangsbedingungen: $\omega_1' = \omega_2' = 0$
\end{solution}
\end{problem}

\begin{problem}{Lichtbrechung}{7}
Der Brechungsindex der Luft bei Atmosphärendruck und $300\unit{K}$ beträgt $1+\tilde n = 1.0003$. Man nehme an, $n-1 \sim \rho$. Um wie viel Mal dichter müsste die Luft an der Erdoberfläche sein, damit sich ein Lichtstrahl in Meereshöhe um die Erde herum bewegen kann? Die Temperatur der Atmosphäre sei als konstant anzunehmen. Der Luftdruck fällt mit der Höhe exponentiell ab, wobei dieser in der Höhe $\tilde h = 8700\unit{m}$ $\frac 1e$ des Normaldrucks beträgt. Der Radius der Erde beträgt $R=6370\unit{km}$.
\begin{solution}
Wenn die Luftdichte um den Faktor $k$ vergrößert wird, gilt für die Brechzahl der Luft:
\[
n(h) = 1 + k \tilde n  e^{-h / \tilde h}
\]
Nach dem Prinzip von \textsc{Fermat} läuft das Licht immer den lokal schnellsten Weg, die Umlaufzeit des Lichts sollte also in erster Ordnung der Höhe konstant sein. Für diese gilt:
\[
T(h) = \frac{2 \pi (R+h) n(h)}{c},
\]
\[
T'(h) \sim \tdif{}{h} (R+h)(1 + k \tilde n  e^{-h / \tilde h}) = (1 + k \tilde n  e^{-h / \tilde h})+(R+h)(k (-1/\tilde h) \tilde n  e^{-h / \tilde h})
\]
\[
0 = \left. T'(h) \right|_{h=0} \sim (1 + k \tilde n)+R (k (-1/\tilde h) \tilde n)
\]
\[
k = \frac{1}{\tilde n (R/ \tilde h - 1)} \approx 4.6
\]
\end{solution}
\end{problem}

\begin{problem}{Bleistift mit Unschaerfe}{8}
Aufgrund der Heisenbergschen Unschärferelation ist es nicht möglich, einen Bleistift für eine längere Zeit auf der Spitze stehen zu lassen. Wie lange kann es maximal dauern, bis ein Bleistift der Länge $15\unit{cm}$ mit der Masse $5\unit{g}$ um $5^\circ$ von der Vertikallage ausgelenkt wird?
%\begin{solution}
% ?
%\end{solution}
\end{problem}

\begin{problem}{Ueberlichtgeschwindigkeit}{8}
Auf einem Planeten führt ein Außerirdischer Versuche zum freien Fall durch. Er wirft Gegenstände mit der Geschwindigkeit $u$ unter dem Winkel $\alpha$ zur Oberfläche des Planeten. Sie beobachten diese aus einem Raumschiff, das sich in der gleichen Ebene parallel zur Planetenoberfläche mit der Geschwindigkeit $v$ bewegt. Zu Ihrem großen Erstaunen stellen Sie fest, dass der beobachtete Geschwindigkeitsanteil der Gegenstände senkrecht zur Oberfläche des Planeten größer als $c$ ist. Man berechne diesen. Wie groß kann er maximal werden?\\
Den Geheimberichten zufolge beträgt die Anfangsgeschwindigkeit der Gegenstände $u=300 \unit{\frac{m}{s}}$ und der Anfangswinkel $\alpha=\frac{\pi}{4}$. Sie beobachten, dass die Gegenstände nach der Zeit $6\unit{s}$ wieder auf der Planetenoberfläche aufschlagen. Wie groß ist die Gravitationsbeschleunigung an der Oberfläche (sie darf als konstant angenommen werden)?
%\begin{solution}
% ?
%\end{solution}
\end{problem}

\ifx \envfinal \undefined
\def\endd{\end{document}}
\expandafter\endd
\fi
