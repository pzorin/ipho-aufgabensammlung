\ifx \mpreamble \undefined
\documentclass[12pt,a4paper]{article}
\usepackage{answers}
\usepackage{microtype}
\usepackage[left=3cm,top=2cm,bottom=3cm,right=2cm,includehead,includefoot]{geometry}

\usepackage{amsfonts,amsmath,amssymb,amsthm,graphicx}
\usepackage[utf8]{inputenc}
\usepackage[T1]{fontenc}
\usepackage{ngerman}

\usepackage{pstricks}
\usepackage{pst-circ}
\usepackage{pst-plot}
%\usepackage{pst-node}
\usepackage{booktabs}

%Times 10^n
\newcommand{\ee}[1]{\cdot 10^{#1}}
%Units
\newcommand{\unit}[1]{\,\mathrm{#1}}
%Differential d's
\newcommand{\dif}{\mathrm{d}}
\newcommand{\tdif}[2]{\frac{\dif#1}{\dif#2}}
\newcommand{\pdif}[2]{\frac{\partial#1}{\partial#2}}
\newcommand{\ppdif}[2]{\frac{\partial^{2}#1}{\partial#2^{2}}}
%Degree
\newcommand{\degr}{^\circ}
%Degree Celsius (C) symbol
\newcommand{\cel}{\unit{^\circ C}}
% Hinweis
\newcommand{\hinweis}{\emph{Hinweis:} }
% Aufgaben mit Buchstaben numerieren
\newenvironment{abcenum}{\renewcommand{\labelenumi}{(\alph{enumi})} \begin{enumerate}}{\end{enumerate}\renewcommand{\labelenumi}{\theenumi .}}

\title{IPhO-Aufgabensammlung}
\author{Zusammengestellt von Pavel Zorin\\
unter Verwendung der Aufzeichungen von\\
Bastian Hacker, Igor Gotlibovych, Tobias Holder,\\
Patrick Steinmüller,\\
sowie anonymer Quellen}

\ifx \envfinal \undefined % Preview mode
\newcommand{\skizze}[1]{
\begin{figure}
\begin{center}
#1
\end{center}
\end{figure}
}

\renewcommand{\thesection}{}
\renewcommand{\thesubsection}{\arabic{subsection}}

\newenvironment{problem}[2]{
\subsection{#1 \emph{(#2 Punkte)}}
}{}
\newenvironment{solution}{\subsection*{Lösung}}{}
\newenvironment{expsolution}{\subsection*{Lösung}}{}

\else % Final mode

\usepackage{pst-pdf}

\usepackage{hyperref}
\hypersetup{
bookmarks=true,
pdfpagemode=UseNone,
pdfstartview=FitH,
pdfdisplaydoctitle=true,
pdflang=de-DE,
pdfborder={0 0 0}, % No link borders
unicode=true,
pdftitle={IPhO-Aufgabensammlung},
pdfauthor={Pavel Zorin},
pdfsubject={Aufgaben der 3. und 4. Runden der deutschen Auswahl zur IPhO},
pdfkeywords={}
}

\newcommand{\skizze}[1]{
\begin{center}
#1
\end{center}
}

%%%%%%%%%%%%%%%%%%%%%%%%%%%%%%%%%%%%%%%%%%%%%%%%%%%%%%%%%%%%%%%%%%%%%%%%%%%%%%%%%%%%%%%%%%%
%%%%%%%%%%%%%%%%%%%%%%%%%%%%%%%%%%%%%%%%%%%%%%%%%%%%%%%%%%%%%%%%%%%%%%%%%%%%%%%%%%%%%%%%%%%
%
%     Formatierung der Aufgaben/Lösungen
%
%     Anmerkung dazu: nicht-ASCII-Zeichen in Überschriften gehen aus rätselhaften Gründen
%     nicht. Sie werden zwar bei der Aufgabe richtig angezeigt, in die Lösungsdatei wird
%     aber eine unverständliche Sequenz geschrieben, die dann nicht wieder gelesen werden
%     kann.
%
%%%%%%%%%%%%%%%%%%%%%%%%%%%%%%%%%%%%%%%%%%%%%%%%%%%%%%%%%%%%%%%%%%%%%%%%%%%%%%%%%%%%%%%%%%%
%%%%%%%%%%%%%%%%%%%%%%%%%%%%%%%%%%%%%%%%%%%%%%%%%%%%%%%%%%%%%%%%%%%%%%%%%%%%%%%%%%%%%%%%%%%
\renewcommand{\thesubsection}{\arabic{subsection}}
\def\solTitle<#1><#2>{
\renewcommand{\thesubsection}{#1}
\subsection{#2}
\renewcommand{\thesubsection}{\arabic{subsection}}
}

\newenvironment{problem}[2]{
\subsection{#1 \emph{(#2 Punkte)}}
\renewcommand{\Currentlabel}{<\thesubsection><#1>}
}{}
\Newassociation{solution}{Soln}{solutions}
\Newassociation{expsolution}{Soln}{expsolutions}
\renewenvironment{Soln}[1]{
\solTitle #1
}{}

\fi

\def \mpreamble {}
\begin{document}

\else
%\ref{test}
\fi


\begin{problem}{Blumenvase}{4}
Gegeben ist eine zylinderförmige Vase mit dünnem Boden, das eine gleichmäßige Masseverteilung hat und eine Höhe von $h=20\unit{cm}$, ein Volumen von $V=1,0\unit{l}$ hat. Man füllt nun Wasser mit einer Dichte von $\rho_W=1000\unit{\frac{kg}{m^3}}$ mit der Höhe $h_F$ in die Vase und erhält eine Funktion für die Höhe $h_S$ des Schwerpunktes des Gesamtsystems in Abhängigkeit von der Höhe $h_F$. In der Klausur war eine Kurve gezeichnet, von der hier nur drei (ausreichend viele) Punkte angegeben werden, die man auch ablesen konnte: $h_F=0\unit{cm}$ oder $h_F=20\unit{cm}$ ergibt $h_S=10\unit{cm}$ und $h_F=4\unit{cm}$ ergibt $h_S=8\unit{cm}$. Zu bestimmen ist die Masse der Vase.
\end{problem}

\begin{problem}{Energieerzeugung in der Sonne}{6}
Es gab eine Theorie, die besagte, dass die Strahlungsleistung der Sonne auf die Gravitationskontraktion bei der Verringerung des Radius der Sonne zurückzuführen ist. Wegen des Potentials der Sonne wird Energie frei. Die Masseverteilung in der Sonne ist während des gesamten Vorganges homogen
\begin{abcenum}
   \item Bestimmen Sie die deshalb freiwerdende Energie, wenn die Sonne den Radius um 50\% verringert und die Masse konstant bleibt.
  \item Bestimmen Sie die bisherige Lebensdauer der Sonne nach dieser Theorie, wenn man annimmt, dass die Energie nur aus der Gravitationskontraktion kommt, die Masse und die Strahlungsleistung der Sonne konstant sind und der Radius anfänglich sehr groß war und nun den heutigen Radius der Sonne hat.
  \item Begründen Sie warum dieses Modell unplausibel ist.
  \item Tatsächlich kommt die Energie aus der Strahlung aus der Kernfusion, wobei Wasserstoffkerne ($H$) zu Heliumkernen fusionieren. Bestimmen Sie nun die Masse an Wasserstoff, die in der Sonne in jeder Sekunde fusionieren muss, um die heutige Strahlungsleistung aufrechtzuerhalten. Schätzen Sie zudem die Gesamtenergie ab, die frei wird, wenn 10\% der Sonnenmasse fusionieren.
  \item Bestimmen Sie die Lebensdauer der Sonne bei heutiger Leistung unter Verwendung des Modells aus d und vergleichen Sie das Ergebnis mit dem Ergebnis aus b.
\end{abcenum}
\begin{description}
  \item[Strahlungsleistung der Sonne:] $P_{Sol}=3,85\ee{26}\unit{W}$
  \item[Radius der Sonne:] $R_{Sol}=6,96\ee{8}\unit{m}$
  \item[Masse der Sonne:] $M_{Sol}=1,989\ee{30}\unit{kg}$
  \item[Masse eines Heliumkernes:] $m_{He}=6,646\ee{-27}\unit{kg}$
\end{description}
\end{problem}

\begin{problem}{Elektronen-Positronen Annihilation}{5}
  Ein Positron trifft mit $v=0,6\cdot c_0$ auf ein ruhendes Elektron, wobei beide Teilchen annhilieren. Hierbei entstehen zwei Photenen, von denen sich eines senkrecht zur Bewegungsrichung bewegt. Bestimmen Sie den Winkel relativ zur Bewegungsrichtung des zweiten Photons, sowie die beiden Wellenlängen.
\end{problem}

\begin{problem}{Kondensator-Spulenschaltung}{6}
Gegeben ist eine Schaltung, die nur ideale Kondensatoren und Spulen enthält. Zwischen den Punken A und B wird die effektive Spannung mit einem idealen Voltmeter gemessen. Zwischen diesen ist ein Kond. mit der Kapazität $C_1$ geschaltet und parallel hierzu eine Spule mit der Induktivität $L_1$ und parallel hierzu in Reihe eine Spule ($L_2$) und ein Kondensator ($C_1$). Durch A und B fließt ein Wechselstrom mit $I(t)=I_0\sin{\omega t}$
\begin{abcenum}
  \item Geben Sie einen Ausdruck für die effektive Spannung $U_{eff}$in Abhängigkeit von den Parametern an.\\ Geben Sie außerdem das Verhalten von $U_{eff}$ für große und kleine $\omega$ an und bestimmen Sie hierfür Ausdrücke.
\item Die Schaltung hat zwei Resonanzfrequenzen $\omega_1$ und $\omega_2$. Bestimmen Sie die Kapazitäten der Kondensatoren, wenn die Induktivitäten $L_1=10\unit{mH}$ und $L_2=50\unit{mH}$ sind.
\item Man nehme die Kondensatoren als Plattenkondensatoren und die Induktivitäten als lange Spulen an. Wie ändert sich die Resonanzfrequenzen aus b, wenn sich alle geometrischen Abessungen der Spulen um Faktor 10 verringern.
\end{abcenum}

\end{problem}

\begin{problem}{Augenoperation}{5}
Ein stark kurzsichtiger Mensch trägt eine Linse mit einem Dioptrie-Wert von $-10$, was dem Kehrwert der Brennweite in Metern entspricht, womit hier eine Streulinse vorliegt, die negative Brennweiten hat. Die Linse, die man als dünn annehmen kann, befindet sich in einem Abstand von $d=2,0\unit{cm}$ vor dem Auge. In einer OP werden der Person künstliche Linsen eingesetzt, dass diese nun einen Text in $30\unit{cm}$ Entfernung ohne Brille scharf lesen kann. Ermitteln Sie, ob ihr der Text größer erscheint, und wenn ja um welchen Faktor?
\end{problem}

\begin{problem}{Isoliertes Kabel}{4}
Gegeben ist ein Aluminiumleiter (spezifischer Widerstand  $\rho_{Alu}=2,65\ee{-8}\unit{\Omega\cdot m}$) mit dem Durchmesser $d=4,0\unit{mm}$ durch das ein Strom von $50\unit{A}$ fließt. Der Leiter ist mit einer $2\unit{mm}$ dicken PVC-Schicht überzogen, dass eine Wärmeleitfähigkeit von $\lambda_{PVC}=0,17\unit{\frac{W}{m\cdot K}}$. Schätzen Sie die Temperatur des Leiters ab, wenn angenommen wird, dass die Temperatur der Umgebung und auch der Kabeloberfläche $20\cel$ entspricht.
\end{problem}

\ifx \envfinal \undefined
\def\endd{\end{document}}
\expandafter\endd
\fi

 