\ifx \mpreamble \undefined
\documentclass[12pt,a4paper]{article}
\usepackage{answers}
%\usepackage{microtype}
\usepackage[left=3cm,top=2cm,bottom=3cm,right=2cm,includehead,includefoot]{geometry}

\usepackage{amsfonts,amsmath,amssymb,amsthm,graphicx}
\usepackage[utf8]{inputenc}
\usepackage{ngerman}

\usepackage{pstricks}
\usepackage{pst-circ}
\usepackage{pst-plot}

\ifx \envfinal \empty
\usepackage{pst-pdf}
\fi

\usepackage{booktabs}

% Muss als letztes eingebunden werden
%\usepackage[bookmarks=true,bookmarksnumbered,colorlinks=true,pdftitle={IPhO-Aufgaben},pdfstartview=FitH,pdfauthor={Pavel Zorin}]{hyperref}
\usepackage[bookmarks=false,pdftitle={IPhO-Aufgabensammlung},pdfstartview=FitH,pdfauthor={Pavel Zorin}]{hyperref}

%Times 10^n
\newcommand{\ee}[1]{\cdot 10^{#1}}
%Units
\newcommand{\unit}[1]{\,\mathrm{#1}}
%Differential d's
\newcommand{\dif}{\mathrm{d}}
\newcommand{\tdif}[2]{\frac{\dif#1}{\dif#2}}
\newcommand{\pdif}[2]{\frac{\partial#1}{\partial#2}}
\newcommand{\ppdif}[2]{\frac{\partial^{2}#1}{\partial#2^{2}}}
%Degree
\newcommand{\degr}{^\circ}
%Degree Celsius (C) symbol
\newcommand{\cel}{\,^\circ\mathrm{C}}
% Hinweis
\newcommand{\hinweis}{\emph{Hinweis:} }
% Aufgaben mit Buchstaben numerieren
\newenvironment{abcenum}{\renewcommand{\labelenumi}{(\alph{enumi})} \begin{enumerate}}{\end{enumerate}\renewcommand{\labelenumi}{\theenumi .}}
%%%%%%%%%% Skizzen %%%%%%%%%%%%
%\ifx \envfinal \empty
%%% Final
%\else
%%% Vorschau
%\fi

\def \mpreamble {}
\else
%\ref{test}
\fi
\ifx \envfinal \undefined


\newcommand{\skizze}[1]{
\begin{figure}
\begin{center}
#1
\end{center}
\end{figure}
}




%\documentclass[12pt,a4paper]{article}
\newcounter{numlabel}
\setcounter{numlabel}{0}

\newcommand{\problemlabel}{}
\newenvironment{problem}[2]{
\stepcounter{numlabel}
\renewcommand{\problemlabel}{Aufgabe \the\value{numlabel}: #1}
\subsubsection*{\problemlabel \emph{(#2 Punkte)}}
}{}
\newenvironment{solution}{\subsubsection*{\problemlabel}}{}
\newenvironment{expsolution}{\subsubsection*{\problemlabel}}{}

\begin{document}

\fi
\begin{problem}{Ein Rollpendel}{22}
In einer ca. $8\unit{cm}$ großen runden Blechdose ist an einer Seite der Innenwand ein Zusatzgewicht angebracht. Der Schwerpunkt ist somit seitlich verschoben.\\
Materialien:
\begin{itemize}
\item exzentrische Dose
\item Stativstange
\item Stoppuhr
\item Schere
\item Bindfaden
\item Lineal (kann auch als Experimentiergerät dienen)
\item Gewicht $46\unit{g}$
\end{itemize}
Aufgaben:
\begin{abcenum}
\item Bestimmen Sie mit Hilfe von Rollschwingungen das Trägheitsmoment $J$ der exzentrischen Dose bezüglich ihrer Zylinderachse. (Tipp: Energieansatz)\\
Dazu benötigen Sie die Ergebnisse der Teilaufgaben 2. und 3. \emph{(11 Punkte)}
\item Bestimmen Sie den Abstand $s$ des Schwerpunktes der Dose von ihrer Mittelachse. \emph{(7 Punkte)}
\item Bestimmen Sie die Masse $M$ der Dose. \emph{(4 Punkte)}
\end{abcenum}
\hinweis
\begin{itemize}
\item Für kleine Winkel gelten die Näherungen $\sin{\varphi}\approx\varphi$ und $\cos{\varphi}\approx1-\frac{\varphi^2}{2}$.
\item Der steinersche Satz besagt für das Trägheitsmoment eines Körpers um eine um $a$ vom Schwerpunkt entfernte Achse
\[
J=J_s+M\cdot a^2,
\]
wobei $J_s$ das Trägheitmoment bezüglich einer parallel verlaufenden Schwerpunktachse bezeichnet.
\end{itemize}
\begin{expsolution}
\begin{abcenum}
\item Rollschwingungen mit Schwingungsgleichungen
\item Die Dose am Faden pendeln lassen. Einmal mit Schwerunkt oben, einmal unten.
\[
\Rightarrow\qquad s=\frac{\Delta T}{2\pi}\sqrt{g\left(\Delta T^2g-4l\right)}\quad?
\]
\item Das Lineal im Schwerpunkt aufhängen und als Balkenwaage für Festgewicht und Dose verwenden.
\end{abcenum}
\end{expsolution}
\end{problem}

\begin{problem}{Gleichstromblackbox mit 3 Anschluessen}{10}
\skizze{
\psset{unit=0.68cm}
\begin{pspicture}(-4,-1)(4,4)
\pnode(-2,0){A}
\pnode(2,0){B}
\pnode(0,3){C}
\resistor(A)(C){}
\resistor(B)(C){}
\lamp(A)(B){}
\end{pspicture}
}
Materialien:
\begin{itemize}
\item Gleichstrom Blackbox mit 3 Anschlüssen
\item Gleichstrom Netzgerät $0-7,5\unit{V}$
\item 2 Digitalmultimeter
\end{itemize}
In der Blackbox befindet sich zwischen je 2 Anschlüssen genau eine der folgenden 3 Möglichkeiten:\\ Nichts, ein ohmscher Widerstand (insgesammt nur eine Sorte) oder ein Glühlämpchen (nur eine Sorte).
\begin{abcenum}
\item Welche Elemente befinden sich in der Blackbox und wo?
\item Wie groß sind ggf. die Festwiderstände und wie groß ist ggf. die Leistung der Glühbirne bei $7,5\unit{V}$?
\end{abcenum}
\hinweis bei zu hohen Spannungen können die Elemente der Blackbox beschädigt werden!
\begin{expsolution}
Widerstände grob messen (Symmetrie?) $\rightarrow$ einzelne Verbindungen kurzschließen $\rightarrow$ Elemente genau vermessen $\rightarrow$ Zwei Widerstände und ein Lämpchen
\end{expsolution}
\end{problem}

\begin{problem}{Verlustleistung eines Kondensators}{10}
Materialien:
\begin{itemize}
\item Kondensator
\item Widerstand $1\unit{k\Omega}$
\item Wechselstrom Netzgerät $8\unit{V},\; 50\unit{Hz}$
\item 2 Digitalmultimeter
\end{itemize}
Aufgaben:
\begin{abcenum}
\item Welche Kapazität besitzt der Kondensator unter der Annahme, dass es sich um einen idealen Kondensator handelt?
\item Der reale Kondensator besitzt auch einen ohmschen Widerstand. Wie groß ist dieser? Wie groß ist dann die wirkliche Kapazität? Verwenden Sie dazu geeignete Schaltungen und geben diese an! Vergleichen Sie das Ergebnis mit dem aus (a)!
\end{abcenum}
\begin{expsolution}
\begin{abcenum}
\item $C_i=\frac{1}{Z\cdot 2\pi f}$
\item Mit Volt- und Amperemeter misst man den Scheinwiderstand.\\
Beim Kondensator: $Z^2=R_C^2+(\frac{1}{\omega C})^2$.\\
Mit Widerstand in Reihe: $Z_1^2=(R+R_C)^2+(\frac{1}{\omega C})^2$
\[
\Rightarrow\quad R_C=\frac{Z_1^2-Z^2-R^2}{2R}
\]
\[
C=\frac{R}{\pi f \sqrt{2(R^2Z^2+R^2Z_1^2+Z^2Z_1^2)-(R^4+Z^4+Z_1^4)}}
\]
Die Formel für $C$ sieht sehr kompliziert aus, besitzt aber hohe Symmetrie. Die Formel für den rein kapazitiven Widerstand entspräche der Formel für die Höhe in einem Dreieck.
\end{abcenum}
\end{expsolution}
\end{problem}



\begin{problem}{Coladose}{20}
Materialien:
\begin{itemize}
\item 1 Coladose, die in Wasser gerade noch schwimmt
\item Stoppuhr
\item großer Standzylinder, etwa $1\unit{m}$ hoch
\item Messbecher
\item 2 Messzylinder (50 bzw. 250 ml)
\item Wasser ($\rho = 1000 \unit{kg \cdot m^{-3}}$, wer hätte es gedacht)
\item Knete
\item Tücher zum Abtrocknen
\end{itemize}

\begin{abcenum}
\item Man bestimme $g$ mithilfe von Coladose, Lineal und Stoppuhr. Man beachte, dass die Coladose im Aufbau unbedingt sinnvoll zu verwenden ist!
\item Man lasse nun die Coladose vorsichtig in den Standzylinder gleiten, sodass die obere Fläche trocken bleibt. Wenn man die Dose leicht auslenkt, schiwngt diese eine Weile, kommt aber schnell zu Ruhe. Dies liegt an der Reibung, die sich in diesem Fall folgendermaßer beschreiben lässt:
\[
\vec{F}=-b \vec{v}
\]
Man bestimme $b$ auf 2 verschiedene Weisen: indem man eine Schwingung betrachtet und indem man die Dose unter Wasser hält, loslässt und den Aufstiegsvorgang beobachtet.
\item Nun darf die Dose geöffnet werden. Allerdings ist es nicht ratsam, alles gleich auszutrinken, denn nun geht es darum, die Dichte von Cola sowie die Masse der leeren Dose zu bestimmen. Viel Spaß!
\end{abcenum}

% \begin{expsolution}
% 
% \end{expsolution}
\end{problem}



\begin{problem}{Graphit}{6}
Bestimmen Sie mit den angegebenen Materialien möglichst genau den spezifischen Widerstand von Graphit. \hinweis Die Messung des Radiuses des Graphitstabes mit dem Lineal ist sehr ungenau.
\begin{itemize}
\item Graphitmine
\item Lineal
\item 2 Multimeter
\item 1 Batterie mit eingebautem Vorwiderstand
\item Objektträger
\end{itemize}
% \begin{expsolution}
% 
% \end{expsolution}
\end{problem}



\begin{problem}{Kupfer unnd Messing}{14}
Materialien:
\begin{itemize}
\item Eine Kupfer- und eine Messingplatte gleicher Größe und Dicke
\item 1 Stabmagnet
\item Stativmaterial
\item Stoppuhr
\item Fäden
\item Textilklebeband, Knete
\item Karton
\end{itemize}
Kupfer hat einen spezifischen Widerstand von $1.77 \ee{-8} \unit{\Omega m}$. Man bestimme den spezifischen Widerstand von Messing.
% \begin{expsolution}
% 
% \end{expsolution}
\end{problem}



\begin{problem}{Optische Blackbox}{20}
\skizze{
\psset{unit=0.55cm}
\begin{pspicture}(0,-0.5)(10,4.5)
\psline(0,0)(10,0)
\psline{-|}(0,4)(8,4)
\psline{|-}(9,4)(10,4)
\psline[linestyle=dotted](0,0)(0,4)
\psline[linestyle=dotted](10,0)(10,4)
\psarc(3,2){4.47}{333.5}{26.5}
\psarc(11,2){4.47}{153.5}{206.5}
\end{pspicture}
}
Materialien:
\begin{itemize}
\item 1 Blackbox
\item 1 Lichtquelle
\item 1 Meterstab
\item 1 Schirm
\item Wasser (Brechunsgindex $1.33$)
\end{itemize}
\begin{abcenum}
\item Die Blackbox besteht aus einem schwarzen Rohr, im Inneren dessen eine Linse befestigt ist. Beide Enden des Rohres sind mit Glasplatten verschlossen. Man bestimme mit zwei unterschiedlichen optischen Methoden die Brennweite und die Lage der Linse in der Blackbox.
\item An einer Seite der Linse besitzt die Blackbox eine Öffnung, durch die sich eine Hälfte der Blackbox mit Wasser füllen lässt, sodass Wasser nicht an die andere Seite der Linse kommt. Man bestimme die Krümmung der Linse und den Brechungsindex des Linsenmaterials.
\end{abcenum}

\begin{expsolution}
\begin{abcenum}
 \item \textbf{Methode 1:} Die Linse wird ein Stück von der Lampe aufgestellt. Dann wird der Schirm so positioniert, dass eine scharfe Abbildung des Glühwendels enteteht. Nun dreht man die Blackbox um und verschiebt sie so weit, bis die Abbildung wieder scharf ist. Aus der Verschiebung ergibt sich die Linsenposition. Kennt man diese, ergibt sich die Brennweite aus der Abbildungsgleichung $\frac 1 f = \frac 1 b + \frac 1 g$.\\
 \textbf{Methode 2:} Es wird eine scharfe Abbildung des Glühwendels erzeugt. daraufhin verschiebt man die Lampe seitlich. Auch das Bild verschiebt sich seitlich. Der Strahlensatz liefert nun die Position. Die Brennweite ergibt sich wie im ersten Fall.\\
 \textbf{Methode 3:} Man misst die Größe der scharfen Abbildung in Abhängigkeit vom Schirmabstand. Ein geeigneter Graph der Werte zeigt dann die Linsenposition. Die Brennweite ergibt sich wie im ersten Fall.
 \item Die Linsenmacherformel lautet in diesem Fall:
\[
 \frac r f = 2n-(1+1.33)
\]
Die Messung mit und ohne Wasser liefert mit den Daten aus a) zwei verschiedene Gleichungen, die man lösen kann. Unter der Annahme achsennaher Strahlen bewirkt die Brechung an der wassergefüllten Seite eine Verkürzung des Strahlenganges in der Luft um den Faktor $1.33$. Das muss man noch in die Rechnung mit einbeziehen.
\end{abcenum}
\end{expsolution}
\end{problem}



\begin{problem}{Unbekannte Spannungsquelle}{8}
Materialien:
\begin{itemize}
\item Eine Blackbox mit 2 Anschlüssen
\item 1 Amperemeter
\item 1 Widerstand von $620 \unit{k\Omega}$
\item 1 Kondensator
\item 1 Stoppuhr
\end{itemize}
In der Blackbox  mit 2 Anschlüssen befindet sich eine Gleichspannungsquelle. Bestimmen sie die Spannung dieser Spannungsquelle sowie die Kapazität des gegebenen Kondensators.
\begin{expsolution}
Bestimmung der Spannung: Reihenschaltung mit dem Widerstand:
\[
U = R\cdot I
\]
Bestimmung der Kapazität: Laden des Kondensators, dann Zeit stoppen, umpolen und über den Widerstand entladen:
\[
I(t) = I_0\cdot e^{-\frac{t}{R\,C}}
\]
In kurzen Zeitintervallen (z.B. 4s) die Wertepaare $(I|t)$ notieren. Im Diagramm $t(\ln I)$ auftragen und die Steigung $m$ ablesen.
\[
C = -\frac m R \qquad\qquad C \approx 10\unit{\mu F}
\]
\end{expsolution}
\end{problem}



\begin{problem}{Widerstaende und Dioden}{12}
\skizze{
\psset{unit=0.75cm}
\begin{pspicture}(-1,-1)(7,5)
\pnode(0,4){A}
\pnode(3,0){B}
\pnode(6,4){C}
\pnode(3,4){M}
\pnode(3,2){N}
\pnode(3,1){N2}
%Widerstände
\resistor(A)(M){}
\resistor(N2)(N){}
\wire(B)(N2)
%Dioden
\diode(M)(N){}
\diode(M)(C){}
%Anschlüsse
\uput[ul](0,4){$A$}
\pscircle*(A){3pt}
\uput[dl](3,0){$B$}
\pscircle*(B){3pt}
\uput[ur](6,4){$C$}
\pscircle*(C){3pt}
\end{pspicture}
}
Materialien:
\begin{itemize}
\item Eine Blackbox mit 3 Anschlüssen
\item 1 Batterie
\item 1 Multimeter zur Benutzung als Amperemeter und Voltmeter
\item 1 Potentiometer
\end{itemize}
In der Blackbox mit 3 Anschlüssen befinden sich ein oder zwei Widerstände gleichen Typs und eine Anzahl ($\leq 3$) Dioden gleichen Typs.
\begin{abcenum}
\item Bestimmen sie zunächst, welche Bauteile in der Blackbox enthalten und wie diese geschaltet sind.
\item Wie groß ist der Widerstand des verwendeten Widerstandstyps?
\item Bestimmen sie schließlich die Kennlinie des verwendeten Diodentyps.
\end{abcenum}
Eine ausführliche Fehlerrechnung wird nicht verlangt.
\begin{expsolution}
% \skizze{
% \psset{unit=0.75cm}
% \begin{pspicture}(-1,-1)(7,5)
% \pnode(0,4){A}
% \pnode(3,0){B}
% \pnode(6,4){C}
% \pnode(3,4){M}
% \pnode(3,2){N}
% \pnode(3,1){N2}
% %Widerstände
% \resistor(A)(M){}
% \resistor(N2)(N){}
% \wire(B)(N2)
% %Dioden
% \diode(M)(N){}
% \diode(M)(C){}
% %Anschlüsse
% \uput[ul](0,4){$A$}
% \pscircle*(A){3pt}
% \uput[dl](3,0){$B$}
% \pscircle*(B){3pt}
% \uput[ur](6,4){$C$}
% \pscircle*(C){3pt}
% \end{pspicture}
% }
Als erstes legt man an jede der 6 Verbindungsmöglichkeiten die Batterie an und misst qualitativ den Strom. Ergebnis: Strom fließt nur von $A$ aus, und nach $C$ fließt in etwa doppelt so viel wie nach $B$. Daraus erschließt sich die Schaltung.

Nun schließt man die Spannungsquelle an die beiden äußeren Anschlüsse des Potentiometers und greift am Mittelabgriff eine variable Spannung ab.\\

Die Widerstände zwischen $AB$ bzw. $AC$ unterscheiden sich um $R$, falls an der Diode die gleiche Spannung bzw. der gleiche Strom anliegt. Letzteres lässt sich einstellen. Also misst man bei jeweils gleichen Strömen die jeweilige Spannung. Es gilt:
\[
 R = \frac{U_{AB}-U_{AC}}{I}
\]
Größere Genauigkeit wird erreicht, indem man Strom und Spannung bei verschiedenen Werten misst und die Steigung der resultierenden Ursprungsgerade ermittelt.\\

Mit dem bekannten Widerstand $R$ lässt sich nun auch die Kennlinie bestimmen. Dazu trägt man den gemessenen Strom $I$ nach der Diodenspannung
\[
 U_D = U_{AC} -R\cdot I
\]
bei verschiedenen Spannungswerten auf. Das ist die gesuchte Kennlinie.
\end{expsolution}
\end{problem}



\begin{problem}{Diodenbehaftete Blackbox}{12}
\skizze{
\psset{unit=1cm}
\begin{pspicture}(0.5,-0.5)(5.5,2.5)
 \pnode(0.5,1){A}
 \pnode(1,0){U0}
 \pnode(1,1){C}
 \pnode(1,2){D}
 \pnode(5,0){U1}
 \pnode(4,1){M1}
\pnode(3,2){O1}
\pnode(4,2){O2}
\resistor[labeloffset=0](U0)(U1){$500 \unit{\Omega}$}
\resistor[labeloffset=0](C)(M1){$200 \unit{\Omega}$}
\resistor[labeloffset=0](D)(O1){$200 \unit{\Omega}$}
\diode(O1)(O2){}
\wire(O2)(M1)
\pnode(4,1.5){R1}
\pnode(5,1.5){R2}
\diode(R1)(R2){}
\wire(R2)(U1)
\pnode(5,1){E1}
\pnode(5.5,1){E2}
\wire(E1)(E2)
% \capacitor(A)(X){$C$}
 \wire(A)(C)
 \wire(U0)(C)
 \wire(C)(D)
% \tension(A)(B){$U(t)$}
\end{pspicture}
}
Materialien:
\begin{itemize}
\item Einstellbare Spannungsquelle mit einem vorgeschalteten Widerstand (bestehen aus einer 9V-Batterie und einem Potentiometer)
\item 2 Multimeter
\item 1 Blackbox
\end{itemize}
Die Blackbox (s. Abb.) enthält einen oder mehrere Widerstände und eine oder mehrere Dioden gleichen Typs. Man bestimme den möglichst einfachen Aufbau der Blackbox, der dem tatsächlichen äquivalent ist.
% \begin{expsolution}
% 
% \end{expsolution}
\end{problem}



\begin{problem}{Halbleiterwunderland}{8}
Materialien:
\begin{itemize}
\item Einstellbare Spannungsquelle aus der letzten Aufgabe
\item 2 LEDs zur Stromanzeige
\item 4 verschiedene Blackboxen jeweils mit 3 Anschlüssen, die nur Halbleiterbauelemente enthalten
\end{itemize}
Man bestimme die aquivalenten Schaltungen.\\
Tatsächlich eingebaut waren:
\begin{itemize}
\item Ein $npn$-Transistor
\item Ein $pnp$-Transistor
\item \ 
\begin{pspicture}(-0.5,-0.3)(2.5,0.3)
\pnode(0,0){A}
\pnode(1,0){B}
\pnode(2,0){C}
\diode(A)(B){}
\diode(C)(B){}
\pscircle*(A){2\pslinewidth}
\pscircle*(B){2\pslinewidth}
\pscircle*(C){2\pslinewidth}
\end{pspicture}
\item wie vorher, nur die Anschlussrichtung der Dioden wurde umgekehrt
\end{itemize}
% \begin{expsolution}
% 
% \end{expsolution}
\end{problem}



\begin{problem}{Totalreflexion}{20}
Materialien:
\begin{itemize}
\item Glyzerin
\item Wasser
\item Spiegel
\item 3 Objektträger
\item Lineal
\item Pappstreifen
\item Stativmaterial
\item Klebeband
\end{itemize}
Man bestimme den Totalreflexionswinkel an der Wasser-Glyzerin-Grenze.\\
\hinweis Aus 2 Objektträgern und einer Flüssigkeit lässt sich ein dünnes Prisma basteln.
% \begin{expsolution}
% 
% \end{expsolution}
\end{problem}



% \begin{problem}{}{}
% 
% \begin{expsolution}
% 
% \end{expsolution}
% \end{problem}

\ifx \envfinal \undefined
\end{document}
\fi