\ifx \mpreamble \undefined
\documentclass[12pt,a4paper]{article}
\usepackage{answers}
\usepackage{microtype}
\usepackage[left=3cm,top=2cm,bottom=3cm,right=2cm,includehead,includefoot]{geometry}

\usepackage{amsfonts,amsmath,amssymb,amsthm,graphicx}
\usepackage[utf8]{inputenc}
\usepackage[T1]{fontenc}
\usepackage{ngerman}

\usepackage{pstricks}
\usepackage{pst-circ}
\usepackage{pst-plot}
%\usepackage{pst-node}
\usepackage{booktabs}

%Times 10^n
\newcommand{\ee}[1]{\cdot 10^{#1}}
%Units
\newcommand{\unit}[1]{\,\mathrm{#1}}
%Differential d's
\newcommand{\dif}{\mathrm{d}}
\newcommand{\tdif}[2]{\frac{\dif#1}{\dif#2}}
\newcommand{\pdif}[2]{\frac{\partial#1}{\partial#2}}
\newcommand{\ppdif}[2]{\frac{\partial^{2}#1}{\partial#2^{2}}}
%Degree
\newcommand{\degr}{^\circ}
%Degree Celsius (C) symbol
\newcommand{\cel}{\unit{^\circ C}}
% Hinweis
\newcommand{\hinweis}{\emph{Hinweis:} }
% Aufgaben mit Buchstaben numerieren
\newenvironment{abcenum}{\renewcommand{\labelenumi}{(\alph{enumi})} \begin{enumerate}}{\end{enumerate}\renewcommand{\labelenumi}{\theenumi .}}

\title{IPhO-Aufgabensammlung}
\author{Zusammengestellt von Pavel Zorin\\
unter Verwendung der Aufzeichungen von\\
Bastian Hacker, Igor Gotlibovych, Tobias Holder,\\
Patrick Steinmüller,\\
sowie anonymer Quellen}

\ifx \envfinal \undefined % Preview mode
\newcommand{\skizze}[1]{
\begin{figure}
\begin{center}
#1
\end{center}
\end{figure}
}

\renewcommand{\thesection}{}
\renewcommand{\thesubsection}{\arabic{subsection}}

\newenvironment{problem}[2]{
\subsection{#1 \emph{(#2 Punkte)}}
}{}
\newenvironment{solution}{\subsection*{Lösung}}{}
\newenvironment{expsolution}{\subsection*{Lösung}}{}

\else % Final mode

\usepackage{pst-pdf}

\usepackage{hyperref}
\hypersetup{
bookmarks=true,
pdfpagemode=UseNone,
pdfstartview=FitH,
pdfdisplaydoctitle=true,
pdflang=de-DE,
pdfborder={0 0 0}, % No link borders
unicode=true,
pdftitle={IPhO-Aufgabensammlung},
pdfauthor={Pavel Zorin},
pdfsubject={Aufgaben der 3. und 4. Runden der deutschen Auswahl zur IPhO},
pdfkeywords={}
}

\newcommand{\skizze}[1]{
\begin{center}
#1
\end{center}
}

%%%%%%%%%%%%%%%%%%%%%%%%%%%%%%%%%%%%%%%%%%%%%%%%%%%%%%%%%%%%%%%%%%%%%%%%%%%%%%%%%%%%%%%%%%%
%%%%%%%%%%%%%%%%%%%%%%%%%%%%%%%%%%%%%%%%%%%%%%%%%%%%%%%%%%%%%%%%%%%%%%%%%%%%%%%%%%%%%%%%%%%
%
%     Formatierung der Aufgaben/Lösungen
%
%     Anmerkung dazu: nicht-ASCII-Zeichen in Überschriften gehen aus rätselhaften Gründen
%     nicht. Sie werden zwar bei der Aufgabe richtig angezeigt, in die Lösungsdatei wird
%     aber eine unverständliche Sequenz geschrieben, die dann nicht wieder gelesen werden
%     kann.
%
%%%%%%%%%%%%%%%%%%%%%%%%%%%%%%%%%%%%%%%%%%%%%%%%%%%%%%%%%%%%%%%%%%%%%%%%%%%%%%%%%%%%%%%%%%%
%%%%%%%%%%%%%%%%%%%%%%%%%%%%%%%%%%%%%%%%%%%%%%%%%%%%%%%%%%%%%%%%%%%%%%%%%%%%%%%%%%%%%%%%%%%
\renewcommand{\thesubsection}{\arabic{subsection}}
\def\solTitle<#1><#2>{
\renewcommand{\thesubsection}{#1}
\subsection{#2}
\renewcommand{\thesubsection}{\arabic{subsection}}
}

\newenvironment{problem}[2]{
\subsection{#1 \emph{(#2 Punkte)}}
\renewcommand{\Currentlabel}{<\thesubsection><#1>}
}{}
\Newassociation{solution}{Soln}{solutions}
\Newassociation{expsolution}{Soln}{expsolutions}
\renewenvironment{Soln}[1]{
\solTitle #1
}{}

\fi

\def \mpreamble {}
\begin{document}

\else
%\ref{test}
\fi


%\subsection*{2006 -- 3. Runde -- Theoretische Klausur II (30.1.06)}

\begin{problem}{Temperaturmessung}{3}
\skizze{
\begin{pspicture}(-1.5,-2.4)(3,1)
\psarc(0,0){1}{-80}{260}

\psline[linearc=0.2,linecolor=white,fillstyle=hlines](0.2,-0.98)(0.2,-1.8)(2.1,-1.8)(2.1,-0.98)(2.5,-0.98)(2.5,-2.2)(-0.2,-2.2)(-0.2,-0.98)

\psline[linearc=0.2](0.2,-0.98)(0.2,-1.8)(2.1,-1.8)(2.1,0.7)
\psline[linearc=0.2](-0.2,-0.98)(-0.2,-2.2)(2.5,-2.2)(2.5,0.7)
\psline(-0.2,-0.98)(0.2,-0.98)
\psline(2.1,-0.98)(2.5,-0.98)
\psline[linestyle=dashed](0.2,-0.98)(2.1,-0.98)
\end{pspicture}
}
Eine Glaskugel mit einem Volumen von $7\unit{l}$ ist mit Luft bei einer Temperatur
von $27\cel$ gefüllt. Das anschließende U-förmige Rohr mit einem Querschnitt von $10\unit{cm^2}$
ist mit Quecksilber gefüllt, so dass dessen Höhe in beiden Schenkeln gleich ist. Der Außendruck
beträgt $760\unit{mmHg}$. Die Luft wird nun erwärmt, so dass der Quecksilberspiegel im rechten Rohr
um $5\unit{mm}$ ansteigt.\\
Welche Temperatur hat die Luft in der Kugel?
\begin{solution}
% ?
Die neue Temperatur beträgt $31.2\cel$
\end{solution}
\end{problem}


\begin{problem}{Bremsmanoever}{6}
Eine Raumsonde umkreist mit der Geschwindigkeit $u$ einen Planeten mit Masse $M$.
Während einer sehr kurzen Zeit wird die Sonde um $\Delta u_1$ in Flugrichtung abgebremst.
Im zum Planeten gegenüberliegenden Punkt wird die Sonde nochmals um $\Delta u_2$ abgebremst,
so dass sie sich wieder auf einer Kreisbahn bewegt.
\begin{abcenum}
\item Bestimmen Sie $\Delta u_2$.
\item Bestimmen Sie $u/\tilde u$, wenn $\tilde u$ die Geschwindigkeit auf der neuen Kreisbahn ist. Ist das Verhältnis größer, kleiner oder gleich 1?
\end{abcenum}
\begin{solution}
Wähle Einheiten so, dass die Masse der Sonde und $G$ gleich $1$ sind.
Für eine Kreisbahn gilt dann (Kräftegleichgewicht)
\[
u^2 / R = M / R^2.
\]
Drehimpuls des Satelliten bzgl. des Planetenzentrums ist zu Beginn also
\[
L_0 = u_0 R_0 = M / u_0,
\]
Energie, mit Nullpunkt bei Ruhe im Unendlichen,
\[
E_0 = u_0^2 / 2 - M / R_0 = - u_0^2 / 2.
\]
Nach dem ersten Manöver betragen die Erhaltungsgrößen noch
\[
L_1 = (u_0 - \Delta u_1) R_0 = (u_0 - \Delta u_1) M / u_0^2,
\]
\[
E_1 = (u_0 - \Delta u_1)^2 / 2 - M / R_0 = (u_0 - \Delta u_1)^2 / 2 - u_0^2.
\]
Man bemerke, dass die Geschwindigkeit des Satelliten weiterhin tangential zu seiner ursprünglichen Bahn gerichtet ist. Aus Symmetriegründen wird dies nach einer halben Umdrehung ebenfalls der Fall sein. Die Geschwindigkeit und Abstand zum Zentrum des Planeten zu diesem Zeitpunkt müssen also folgendes erfüllen:
\[
L_1 = L_2 = u_2 R_2,
\]
\[
E_1 = E_2 = u_2^2 / 2 - M / R_2,
\]
also
\[
u_2 = M/L \pm \sqrt{M^2/L^2 + 2E}, \quad R_2 = L_1/u_2.
\]
Nun beachte noch dass $u_2 \neq u_1$ ist, man wähle also die Lösung mit Plus. Damit die Sonde wieder eine Kreisbahn erreicht, muss nach der anschließenden Bremsung die schon erwähnte Kreisbahngleichung
\[
(u_2 - \Delta u_2)^2 / R_2 = M / R_2^2
\]
gelten. Diese besitzt wiederum die Lösungen
\[
\Delta u_2 = u_2 \pm \sqrt{u_2 M / L_1},
\]
wobei die Lösung mit Plus dem unrealistischen Umkehren der Rotationsrichtung entspricht. Für die Lösung mit Minus bekommt man nach einer einfachen aber aufwendigen Rechnung
\[
\Delta u_2=\frac{(u^2+2u\Delta u_1-\Delta u_1^2)-u\sqrt{u^2+2u\Delta u_1-\Delta u_1^2}}{u-\Delta u_1},
\]
und für das Verhältnis der Geschwindigkeiten
\[
\frac{u}{\tilde{u}}=\frac{u-\Delta u_1}{\sqrt{u^2+2u\Delta u_1-\Delta u_1^2}} < 1.
\]
\end{solution}
\end{problem}


\begin{problem}{Polarisatoren}{4}
Zwischen zwei um $\frac{\pi}{2}$ gegeneinander verdrehten Polarisatoren werden $k=0,1,2,\dots$ weitere Polarisatoren so aufgestellt, dass jeder Polarisator gegenüber dem vorigen um $\frac{\pi}{2(k+1)}$ verdreht ist.

Bestimmen Sie den Anteil $\alpha$ des auf den ersten Polarisator einfallenden unpolarisierten Lichts, welcher vom System durchgelassen wird.

Betrachten Sie insbesondere die Grenzfälle $k=0$ und $k \to \infty$.
\begin{solution}
Das allgemeine Ergebnis lautet
\[
\alpha(k)=\frac{1}{2}\left( \cos^2\frac{\pi}{2(k+1)} \right)^{k+1}.
\]
Im Fall $k=0$ bekommt man $\alpha(0)=0$. Im Fall $k \to \infty$ kann man die für konvergente Folgen $(s_{n})_{n}$ gültige Formel
\[
\lim_{n \to \infty} (1 + s_{n}/n)^{n} = \exp ( \lim_{n \to \infty} s_{n})
\]
benutzen, die unmittelbar $\lim_{k\to\infty} \alpha(k) = \frac{1}{2}$ liefert.
\end{solution}
\end{problem}


\begin{problem}{Manipulation von Elektronen}{5}
\skizze{
  \begin{pspicture}(-0.5,-1.5)(5,1.5)
\psline{->}(0,-1.3)(0,1.3)\uput[l](0,1){$\rho$}
\psline{->}(-0.2,0)(4.5,0)\uput[d](4.5,0){$x$}
\psplot[linewidth=0.5pt,plotpoints=400, arrows=->]{1}{3.5}{1 x sqrt div}\uput[ur](2,0.7){$- \vec{E}$}
\psplot[linewidth=0.5pt,plotpoints=300, arrows=->]{0.25}{3.5}{0.5 x sqrt div}
\psplot[linewidth=0.5pt,plotpoints=200, arrows=->]{0.0625}{3.5}{0.25 x sqrt div}
\psplot[linewidth=0.5pt,plotpoints=100, arrows=->]{0.015625}{3.5}{0.125 x sqrt div}
\psplot[linewidth=0.5pt,plotpoints=100, arrows=->]{0.015625}{3.5}{0.125 neg x sqrt div}
\psplot[linewidth=0.5pt,plotpoints=200, arrows=->]{0.0625}{3.5}{0.25 neg x sqrt div}
\psplot[linewidth=0.5pt,plotpoints=300, arrows=->]{0.25}{3.5}{0.5 neg x sqrt div}
\psplot[linewidth=0.5pt,plotpoints=400, arrows=->]{1}{3.5}{1 neg x sqrt div}
\psline[linewidth=0.5pt]{|-|}(3.8,0)(3.8,0.3)\uput[r](3.8,0.15){$r$}
\end{pspicture}
}
\begin{abcenum}
\item Elektronen durchlaufen eine Beschleunigungsspannung von $U_\mathrm{B}$ und kommen dann
unter einem Winkel $\varphi$ durch das Loch in einer der Platten in das Feld eines Plattenkondensators,
an dem eine Spannung $U_\mathrm{K}$ anliegt.\\
Welche Geschwindigkeit hat das Elektron, wenn es durch ein zweites Loch den Kondensator wieder verlässt?
\item In einem bestimmten ladungsfreien Bereich herrscht ein radialsymmetrisches elektrisches Feld. Die $x$-Komponente des Feldes hängt dabei nur von $x$ ab und beträgt $E_x(x) = -2 a x$ mit $a=2\ee{6}\unit{V/m^2}$. Man bestimme die Radialkomponente des Feldes im achsennahen Bereich und insbesondere im Abstand $r=3\unit{mm}$ von der $x$-Achse.
\end{abcenum}

\begin{solution}
\begin{abcenum}
\item Die Geschwindigkeit beträgt (nicht-relativistisch) $\sqrt{\frac{2e(U_\mathrm{B}+U_\mathrm{K})}{m}}$, falls das Elektron die gegen\-über\-liegende Platte erreicht, und $\sqrt{\frac{2eU_\mathrm{B}}{m}}$, falls das Elektron im Kondensator umkehrt und wieder durch die gleiche Platte geht. Dies ist für $\cos\varphi<\sqrt{\frac{U_\mathrm{K}}{U_\mathrm{B}}}$
der Fall, falls das $U_\mathrm{K}$ so anliegt, dass die Platte, durch die das Elektron eintritt, positiv geladen ist.\\
\item Divergenzfreiheit des statischen elektrischen Feldes in Abwesenheit von Ladungen liefert
\[
0 = \nabla \vec E = \pdif{}{x} E_x + \frac1\rho \pdif{}{\rho} (\rho E_\rho) = -2 a + E_\rho' + \frac{1}{\rho} E_\rho.
\]
Die Lösung dieser Gleichung mit $E_\rho \to 0$ bei $\rho \to 0$ ist $E_\rho(\rho) = a \rho$, die exakte Lösung stimmt also mit der achsennahen Näherung überein. Im Abstand $r$ bekommt man schließlich
\[
E_\rho(r)=6\ee{3}\unit{V/m}.
\]
\end{abcenum}
\end{solution}
\end{problem}


\begin{problem}{Ionisiertes Helium}{6}
Das ionisierte Helium He$^+$ verhält sich wasserstoffähnlich. Es hat die Energieniveaus
$-\frac{54,4\unit{eV}}{n^2}$ mit $n=(1,2,\ldots)$. Ein He$^+$-Gas wird aus einer bestimmten Richtung
mit Licht der Wellenlängen $24\unit{nm}\leq\lambda\leq50\unit{nm}$ bestrahlt.
\begin{abcenum}
\item Wie viele Absorptionslinien sieht man, wenn man entgegen der Strahlungsrichtung blickt? Welche Wellenlängen haben sie?
\item Welche Emissionslinien beobachtet man entgegen bzw. senkrecht zur Strahlungsrichtung?
\end{abcenum}
\begin{solution}
\begin{abcenum}
\item Man sieht 3 Absorbtionslinien: $24,3\unit{nm}$, $25,6\unit{nm}$, $30,4\unit{nm}$
\item senkrecht 6 Emissionslinien: $24,3\unit{nm}$, $25,6\unit{nm}$, $30,4\unit{nm}$, $121,6\unit{nm}$, $164,1\unit{nm}$, $468,9\unit{nm}$, entgegen: die selben, außer die von (a)
\end{abcenum}
\end{solution}
\end{problem}


\begin{problem}{Fallender Magnet}{7}
Ein Stabmagnet mit dem magnetischen Moment $\mu$ fällt mit dem Nordpol nach unten in einem engen isolierenden Zylinder. Um diesen herum befindet sich ein zweiter isolierender Zylinder mit dem Radius $r$, auf dem in regelmäßigen Abständen $h$ Leiterschleifen aufgewickelt sind, die einen Widerstand von je $R$ haben. Der Magnet hat die Masse $m$ und erreicht nach kurzer Zeit eine konstante Geschwindigkeit $v$, die von $m$, $\mu$, $h$, $R$, $r$, $\mu_0$ abhängt.\\
Wie ändert sich $v$, wenn man jeweils eine der Größen $m$, $\mu$, $h$, $R$, $r$ verdoppelt und die anderen dabei konstant lässt?\\
Die Luftreibung, das Erdmagnetfeld sowie Selbst- und Gegeninduktion der Leiterschleifen können vernachlässigt werden.
\begin{solution}
Die Änderung von $v$ durch die Parameter lässt sich mit einer Einheitenanalyse bestimmen.
\[
\Rightarrow\qquad v \sim \frac{m\cdot g\cdot h\cdot R \cdot r^3}{\mu^2\cdot \mu_0^2}
\]
Alternativ kann man auch die vertikale Komponente des Magnetfeldes des Stabmagneten $B_z(z,r)$ bestimmen, dann hat man für die Spannung an einem Ring im Abstand $z$ vom Magneten
\[
U(z) = \tdif{}{t} \int\limits_{\rho \leq r} B(z,\rho) \dif A = v \tdif{}{z} \int\limits_{\rho \leq r} B(z,\rho) \dif A
\]
Wenn man nun eine Zeit $\Delta t = h/v$ und die zugehörigen Energieverluste an allen Leiterschleifen betrachtet, bekommt man
\[
\Delta E = \sum_{i=-\infty}^{+\infty} \int\limits_{t=0}^{\Delta t} \frac{U^2(z_i(t))}{R} \dif t
= \frac{1}{Rv} \int\limits_{z=-\infty}^{+\infty} U^2(z) \dif z = mgh
\]
Es reicht dann bei $\Delta E$ einen der Terme (sinnvollerweise den einfachsten) zu integrieren um die Exponenten der Parameter zu bekommen.

\end{solution}
\end{problem}

\ifx \envfinal \undefined
\def\endd{\end{document}}
\expandafter\endd
\fi
