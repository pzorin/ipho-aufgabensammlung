\ifx \mpreamble \undefined
\documentclass[12pt,a4paper]{article}
\usepackage{answers}
\usepackage{microtype}
\usepackage[left=3cm,top=2cm,bottom=3cm,right=2cm,includehead,includefoot]{geometry}

\usepackage{amsfonts,amsmath,amssymb,amsthm,graphicx}
\usepackage[utf8]{inputenc}
\usepackage[T1]{fontenc}
\usepackage{ngerman}

\usepackage{pstricks}
\usepackage{pst-circ}
\usepackage{pst-plot}
%\usepackage{pst-node}
\usepackage{booktabs}

%Times 10^n
\newcommand{\ee}[1]{\cdot 10^{#1}}
%Units
\newcommand{\unit}[1]{\,\mathrm{#1}}
%Differential d's
\newcommand{\dif}{\mathrm{d}}
\newcommand{\tdif}[2]{\frac{\dif#1}{\dif#2}}
\newcommand{\pdif}[2]{\frac{\partial#1}{\partial#2}}
\newcommand{\ppdif}[2]{\frac{\partial^{2}#1}{\partial#2^{2}}}
%Degree
\newcommand{\degr}{^\circ}
%Degree Celsius (C) symbol
\newcommand{\cel}{\unit{^\circ C}}
% Hinweis
\newcommand{\hinweis}{\emph{Hinweis:} }
% Aufgaben mit Buchstaben numerieren
\newenvironment{abcenum}{\renewcommand{\labelenumi}{(\alph{enumi})} \begin{enumerate}}{\end{enumerate}\renewcommand{\labelenumi}{\theenumi .}}

\title{IPhO-Aufgabensammlung}
\author{Zusammengestellt von Pavel Zorin\\
unter Verwendung der Aufzeichungen von\\
Bastian Hacker, Igor Gotlibovych, Tobias Holder,\\
Patrick Steinmüller,\\
sowie anonymer Quellen}

\ifx \envfinal \undefined % Preview mode
\newcommand{\skizze}[1]{
\begin{figure}
\begin{center}
#1
\end{center}
\end{figure}
}

\renewcommand{\thesection}{}
\renewcommand{\thesubsection}{\arabic{subsection}}

\newenvironment{problem}[2]{
\subsection{#1 \emph{(#2 Punkte)}}
}{}
\newenvironment{solution}{\subsection*{Lösung}}{}
\newenvironment{expsolution}{\subsection*{Lösung}}{}

\else % Final mode

\usepackage{pst-pdf}

\usepackage{hyperref}
\hypersetup{
bookmarks=true,
pdfpagemode=UseNone,
pdfstartview=FitH,
pdfdisplaydoctitle=true,
pdflang=de-DE,
pdfborder={0 0 0}, % No link borders
unicode=true,
pdftitle={IPhO-Aufgabensammlung},
pdfauthor={Pavel Zorin},
pdfsubject={Aufgaben der 3. und 4. Runden der deutschen Auswahl zur IPhO},
pdfkeywords={}
}

\newcommand{\skizze}[1]{
\begin{center}
#1
\end{center}
}

%%%%%%%%%%%%%%%%%%%%%%%%%%%%%%%%%%%%%%%%%%%%%%%%%%%%%%%%%%%%%%%%%%%%%%%%%%%%%%%%%%%%%%%%%%%
%%%%%%%%%%%%%%%%%%%%%%%%%%%%%%%%%%%%%%%%%%%%%%%%%%%%%%%%%%%%%%%%%%%%%%%%%%%%%%%%%%%%%%%%%%%
%
%     Formatierung der Aufgaben/Lösungen
%
%     Anmerkung dazu: nicht-ASCII-Zeichen in Überschriften gehen aus rätselhaften Gründen
%     nicht. Sie werden zwar bei der Aufgabe richtig angezeigt, in die Lösungsdatei wird
%     aber eine unverständliche Sequenz geschrieben, die dann nicht wieder gelesen werden
%     kann.
%
%%%%%%%%%%%%%%%%%%%%%%%%%%%%%%%%%%%%%%%%%%%%%%%%%%%%%%%%%%%%%%%%%%%%%%%%%%%%%%%%%%%%%%%%%%%
%%%%%%%%%%%%%%%%%%%%%%%%%%%%%%%%%%%%%%%%%%%%%%%%%%%%%%%%%%%%%%%%%%%%%%%%%%%%%%%%%%%%%%%%%%%
\renewcommand{\thesubsection}{\arabic{subsection}}
\def\solTitle<#1><#2>{
\renewcommand{\thesubsection}{#1}
\subsection{#2}
\renewcommand{\thesubsection}{\arabic{subsection}}
}

\newenvironment{problem}[2]{
\subsection{#1 \emph{(#2 Punkte)}}
\renewcommand{\Currentlabel}{<\thesubsection><#1>}
}{}
\Newassociation{solution}{Soln}{solutions}
\Newassociation{expsolution}{Soln}{expsolutions}
\renewenvironment{Soln}[1]{
\solTitle #1
}{}

\fi

\def \mpreamble {}
\begin{document}

\else
%\ref{test}
\fi


%\subsection*{2007 -- 3. Runde -- Theoretische Klausur I (27.01.2007)}
\begin{problem}{Magnetische Linse}{3,5}
Durch ein Zylinder der Länge $L$ mit Radius $R$ fließt ein konstanter homogener Strom $I$ parallel zur Achse. Man zeige, dass ein parallel zur Achse in den Zylinder eintretender Teilchenstrahl, bestehend aus Teilchen mit positiver Ladung $q$, fokussiert wird. Man finde die Brennweite dieser Linse.
\begin{solution}
Feld innerhalb des Zylinders:
\[
B(r)=\frac{\mu_0 I}{2 \pi R^2}r
\]
Der auf ein Teilchen übertragene Impuls:
\[
p=Ft=(vBq)(\frac Lv)=LBq=\frac{\mu_0 I L q}{2 \pi R^2}r
\]
\[
f=r\frac{v}{\frac pm}=mv \cdot \frac{2 \pi R^2}{q \mu_0 I L}
\]
\end{solution}
\end{problem}

\begin{problem}{Fischbeobachtung}{6,5}
Ein Fisch schwimmt in einem Aquarium in $10 \unit{cm}$ Tiefe. Ein Biologe beobachtet ihn durch eine aus einer bikonvexen Linse mit Krümmungsradien $25 \unit{cm}$ bestehende Lupe, die er horizontal $5 \unit{cm}$ über der Wasseroberfäche hält. Der Fisch befindet sich dabei auf der optischen Achse der Lupe.
\begin{abcenum}
\item Man finde die scheinbare Position des Fisches.
\item Nun wird die untere Fläche der Lupe in Wasser eingetaucht. Wie verändert sich die scheinbare Position des Fisches?
\end{abcenum}
Der Brechungsindex des Glases sei dabei $1.5$, der des Wassers $\frac43$.
\begin{solution}
\begin{abcenum}
\item
\[
h'=\frac{h}{n_{H_2O}}
\]
\[
\frac 1f = (n-1)\left( \frac 1{R_1}+\frac 1{R_2} \right)
\]
\[
b=-25 \unit{cm}
\]
\item
Die Abbildungsgleichung einer kugelförmigen Übergangsfläche mit Radius $R$ zwischen Medien der optischen Dichten $n_1$ bzw. $n_2$ lautet
\[
b = \frac{R}{\frac{n_1}{n_2} \left( 1+\frac R g \right) -1}
\]
Wenn man diese Gleichung zwei Mal hintereinander anwendet, bekommt man für die scheinbare Tiefe $9.73 \unit{cm}$. Dabei ist zu beachten, dass an der zweiten Oberfläche die Krümmung bzw. der Krümmungsradius negativ ist.
\end{abcenum}
\end{solution}
\end{problem}

\begin{problem}{Schwingung}{3}
Ein Gefäß mit Volumen $V$ endet mit einem Rohr (Radius $r$). Dieses Rohr wird mit einem Ball verschlossen, dessen Radius ebenfalls $r$ und dessen Masse $m$ beträgt. Nach dem Einstellen der Ruhelage wird der Ball ausgelenkt. Man finde die Schwingungsfrequenz.
\begin{solution}
\[
pV^\gamma =\mathrm{const}
\]
\[
\dif p V^\gamma + p \gamma V^{\gamma-1} \dif V=0
\]
\[
F=A \Delta p
\]
\[
f=\frac{r^2}{2}\left( \frac{\gamma (p_0+\frac{mg}{\pi r^2})}{Vm} \right)^{\frac 12}
\]
\end{solution}
\end{problem}

\begin{problem}{Komet}{5,5}
Ein Komet auf einer parabolischen Bahn nähert sich der Sonne. Im sonnennächsten Punkt beträgt der Abstand zur Sonne $\frac R3$, wobei $R$ den Radius der Erdumlaufbahn bezeichnet. Man finde die Zeit, innerhalb deren der Abstand des Kometen zur Sonne kleiner als der der Erde zur Sonne ist. Masse der Sonne $m_S$ und $R$ sind gegeben.\\
\hinweis das Integral $\int \frac{x \dif x}{\sqrt{x-a}}=\frac 23 (x+2a) \sqrt{x-a}$, $x>a$, könnte helfen.
\begin{solution}
Energieerhaltung:
\[
\frac 12 m_K v^2=G \frac{m_K m_S}{r}
\]
Drehimpulserhaltung:
\[
v_T r=\mathrm{const}=\frac{R}{3}\sqrt{\frac{6 G m_S}{R}}
\]
Radialgeschwindigkeit:
\[
v_R=\sqrt{v^2-v_T^2}
\]
Zeit innerhalb der Erdumlaufbahn:
\[
T=2 \int\limits_\frac R3^R \frac{\dif r}{v_R}=\frac{10 R}{9} \sqrt{\frac{4 R}{3 G m_S}}
\]
\end{solution}
\end{problem}

\begin{problem}{Reflexion am bewegten Spiegel}{6,5}
Ein vertikaler Spiegel bewegt sich nach rechts mit der Geschwindigkeit $v=\frac c2$. Ein Lichtstrahl trifft darauf unter dem Winkel $\alpha=\frac \pi 6$ zur Normalen. Man finde den Winkel, den der austretende Strahl mit der Normalen einschließt. Wie verändert sich die Frequenz des Lichtes bei der Reflexion?
\begin{solution}
siehe eine Aufgabe aus der 4. Runde 2006:
\[
\cos\beta=\frac{(1+(\frac{v}{c})^2)\cos\alpha-2\frac{v}{c}}{1-2\frac{v}{c}\cos\alpha+(\frac{v}{c})^2}
\]
Lösungsweg: 2x Lorentztransformation anwenden. Erg.: $\beta=10.2^\circ$.
\[
f' \approx 2.8 f
\]
\end{solution}
\end{problem}

\begin{problem}{Kugelkondensator}{5}
\skizze{
\psset{unit=0.75cm}
\begin{pspicture}(-3.75,-2.6)(3.75,2.6)
\pscircle[dimen=outer](0,0){2.5}
\psarc{|-|}(0,0){2}{5.0}{355.0}
\pscircle[dimen=outer](0,0){1.5}
\psline(1.5,0.0)(2.5,0.0)
\rput[0,0]{30}(0,0){\psline[linewidth=1pt]{->}(0,0)(2.5,0)}
\rput[0,0]{330}(0,0){\psline[linewidth=1pt]{->}(0,0)(2.0,0)}
\rput[0,0]{300}(0,0){\psline[linewidth=1pt]{->}(0,0)(1.5,0)}
\rput{30}(0,0){\rput{-30}(1,0.25){$a$}}
\rput[0,0]{330}(0,0){\rput{-330}(1,0.25){$b$}}
\rput[0,0]{300}(0,0){\rput{-300}(1,-0.35){$d$}}
\end{pspicture}
}
Ein Kugelkondensator, bestehend aus 3 metallischen Kugeln, von denen die äußere und die innere leitend verbunden sind, hat die in der Skizze angegebenen Maße. Man bestimme die Kapazität des Kondensators.
\begin{solution}
\[
C=4\pi \varepsilon_0 \left( \frac{ab}{b-a}+\frac{bd}{d-b} \right)=4 \pi \varepsilon_0 \frac{b^2 (d-a)}{(d-b)(b-a)}
\]
\end{solution}
\end{problem}

\ifx \envfinal \undefined
\def\endd{\end{document}}
\expandafter\endd
\fi

