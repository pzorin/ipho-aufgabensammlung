\ifx \mpreamble \undefined
\documentclass[12pt,a4paper]{article}
\usepackage{answers}
%\usepackage{microtype}
\usepackage[left=3cm,top=2cm,bottom=3cm,right=2cm,includehead,includefoot]{geometry}

\usepackage{amsfonts,amsmath,amssymb,amsthm,graphicx}
\usepackage[utf8]{inputenc}
\usepackage{ngerman}

\usepackage{pstricks}
\usepackage{pst-circ}
\usepackage{pst-plot}

\ifx \envfinal \empty
\usepackage{pst-pdf}
\fi

\usepackage{booktabs}

% Muss als letztes eingebunden werden
%\usepackage[bookmarks=true,bookmarksnumbered,colorlinks=true,pdftitle={IPhO-Aufgaben},pdfstartview=FitH,pdfauthor={Pavel Zorin}]{hyperref}
\usepackage[bookmarks=false,pdftitle={IPhO-Aufgabensammlung},pdfstartview=FitH,pdfauthor={Pavel Zorin}]{hyperref}

%Times 10^n
\newcommand{\ee}[1]{\cdot 10^{#1}}
%Units
\newcommand{\unit}[1]{\,\mathrm{#1}}
%Differential d's
\newcommand{\dif}{\mathrm{d}}
\newcommand{\tdif}[2]{\frac{\dif#1}{\dif#2}}
\newcommand{\pdif}[2]{\frac{\partial#1}{\partial#2}}
\newcommand{\ppdif}[2]{\frac{\partial^{2}#1}{\partial#2^{2}}}
%Degree
\newcommand{\degr}{^\circ}
%Degree Celsius (C) symbol
\newcommand{\cel}{\,^\circ\mathrm{C}}
% Hinweis
\newcommand{\hinweis}{\emph{Hinweis:} }
% Aufgaben mit Buchstaben numerieren
\newenvironment{abcenum}{\renewcommand{\labelenumi}{(\alph{enumi})} \begin{enumerate}}{\end{enumerate}\renewcommand{\labelenumi}{\theenumi .}}
%%%%%%%%%% Skizzen %%%%%%%%%%%%
%\ifx \envfinal \empty
%%% Final
%\else
%%% Vorschau
%\fi

\def \mpreamble {}
\else
%\ref{test}
\fi
\ifx \envfinal \undefined


\newcommand{\skizze}[1]{
\begin{figure}
\begin{center}
#1
\end{center}
\end{figure}
}




%\documentclass[12pt,a4paper]{article}
\newcounter{numlabel}
\setcounter{numlabel}{0}

\newcommand{\problemlabel}{}
\newenvironment{problem}[2]{
\stepcounter{numlabel}
\renewcommand{\problemlabel}{Aufgabe \the\value{numlabel}: #1}
\subsubsection*{\problemlabel \emph{(#2 Punkte)}}
}{}
\newenvironment{solution}{\subsubsection*{\problemlabel}}{}
\newenvironment{expsolution}{\subsubsection*{\problemlabel}}{}

\begin{document}

\fi

\begin{problem}{Eingespanntes Gummiband}{4}
Gegegeben ist ein Gummiband, das an zwei horizontalen Platten im Abstand von $40\unit{cm}$ verbunden ist und sich bei Dehnung wie eine Feder verhält. Das Gummiband hat eine Federkonstante von $k=1\unit{\frac{N}{cm}}$
\begin{abcenum}
  \item Das Gummi hat eine Ruhelänge von $l_0=40\unit{cm}$. Man zieht an einem Punkt am Gummi im Abstand $x$ zur oberen Platte mit der Kraft von $15\unit{N}$ nach unten. Bestimmen Sie den Abstand $x$ für den $\Delta x$, die Verschiebung dieses Punktes maximal ist, und den Wert der Verschiebung.
  \item Bestimmen Sie diese beiden Werte für den Fall, dass die Ruhelänge des Gummis wesentlich kleiner ist als $40\unit{cm}$.
\end{abcenum}
%\begin{solution}
%\end{solution}
\end{problem}

\begin{problem}{Geladene Kugel am Faden}{4}
Eine kleine geladene Kugel der Masse $m=10\unit{g}$ hängt an einem masselosem und vollständig isolierenden Faden der Länge $l=1,0\unit{m}$. Die Kugel dreht sich mit der Winkelgeschwindigkeit $\omega$ um die Senkrechte Achse, so dass der Faden einen Winkel von $30\deg$ zur Senkrechten hat. Parallel zur Senkrechten und zur Gravitationskraft liegt ein Magnetfeld der Stärke $B=50\unit{mT}$ an.
\begin{abcenum}
  \item Die Differenz der Frequenzen bei Drehung um und gegen den Uhrzeigersinn beträgt $\Delta f=2\ee{-3}\unit{Hz}$. Bestimmen Sie die Ladung der Kugel.
  \item Bestimmen Sie den Mittelwert der beiden Frequenzen.
\end{abcenum}
\end{problem}

\begin{problem}{Oberflächentemperatur von Neptun}{4}
  \begin{abcenum}
    \item Schätzen Sie die Oberflächentemperatur auf dem Neptun unter Verwendung folgender Informationen ab:
      \begin{itemize}
        \item Radius der Erde $R_E=6380\unit{km}$
        \item Radius des Neptuns $R_N=24500\unit{km}$
        \item Umlaufzeit des Neptuns $t_N=165\unit{a}$
        \item Abstand Erde Sonne $a_E=150\ee{6}\unit{km}$
        \item Temperatur auf der Erde $T_E=14\unit{\cel}$
        \item Strahlungsleistung der Sonne auf der Erde $P_S=1400\unit{\frac{W}{m^2}}$
      \end{itemize}
    \item Durch innere Prozesse wird im Neptun additiv Wärme erzeugt, die dem doppelten Wert der Strahlungsleistung der Sonne auf dem Neptun entspricht. Besimmen Sie nun die Oberflächentemperatur auf dem Neptun.
  \end{abcenum}
\end{problem}

\begin{problem}{Aufsteigende Luftmassen}{5,5}
Ein großes und flaches durch die Sonne erhitzes und trockenes Luftpaket steigt in der Athmosphäre auf.
\begin{abcenum}
  \item Nehmen Sie an, dass die Temperatur in den Höhen des Aufstieges konstant ist. Bestimme die Höhe, ab der das Luftpaket nicht mehr ansteigt. (Hinweis: Sie sollten den Außendruck in Abhängigkeit von der Höhe bestimen)
  \item Nehmen Sie nun an, dass die Luft in Bodennähe feucht ist und Wasserdampf enthält, dass auf dem Weg kondensiert und aus dem Luftpaket zu Boden fällt. Begründen Sie, ob das Luftpaket höher oder weniger hoch steigt.
\end{abcenum}
Sie können folgende Werte für ihre Rechnungen verwenden:
\begin{description}
\item[Temperatur der Außenluft:] $T_0=10\unit{\cel}$
\item[Temperatur des Luftpaketes:] $T_1=25\unit{\cel}$
\item[Adiabatenkoeffizient:] $\kappa=1,4\unit{m^3}$
\item[Molare Masse von Luft:] $M_{Luft}=29,0\unit{g mol^{-1}}$
\end{description}
\end{problem}

\begin{problem}{Lichtbrechung in Glasstange}{4}
Gegeben ist ein langer Glaszylinder, der an einem Ende eben ist und am Anderen mit einer Halbkugel, wobei der Punkt an dessen Spitze P sei, des gleichen Radius endet. Es fallen nun parallele und achsennahe Strahlen auf die Glasstange. Die Strahlen, die auf das ebene Ende treffen brechen derart, dass der Brennpunkt dieser Strahlen im Abstand $a=25\unit cm$ hinter P ist. Die Strahlen, die zuerst auf die Halbkugel treffen haben einen Brennpunkt in der Stange im Abstand $b=14\unit cm$ von P.\\
Bestimmen Sie nun den Brennwert $n$ des Materials der Stange.
\end{problem}

\begin{problem}{Ein Wackelschwinger}{7,5}
Gegeben ist ein Quader mit den Seitenlängen $2h$ und $2b$ und $c$. Im folgenden soll die Bewegung/Schwingung des Quaders um zwei Kanten untersucht werden. Der Quader steht hierbei stets auf einer Kante der Länge $c$ und die Schwingung gestaltet sich derart, dass der Quader um die Seite der Länge $2b$ schwingt und die Höhe $2h$ in dem Punkt, in dem beide Kanten auf dem Boden sind senkrecht zum Boden steht.
\begin{abcenum}
  \item Der Quader steht auf einer Kante mit dem Winkel $\alpha$ der Seite der Länge $2h$ zur Senkrechten. Bestimmen Sie das Rückstellmoments des Quaders um diese Kante in Abhängigkeit des Winkels $\alpha$ und skizzieren Sie dieses. Bestimmen Sie zudem das maximale Rückstellmoment und den maximalen Kippwinkel.
  \item Bestimmen Sie das Trägheitsmoment des Quaders um eine Kante.
  \item Geben Sie einen Ausdruck für ein Viertel der Schwingungsperiode um eine Kante in Abhängigkeit vom Startwinkel (man kann kleine Startwinkel annehmen) der Schwingung an. Es handelt sich hierbei um die Zeit vom Start-Kippwinkel bis es auf beiden Kanten steht. Skizziere diese Abhängigkeit.
  \item Sobald der Quader mit beiden Kanten den Boden berührt, kippt er wieder nach oben, so dass diese jetzt um die andere Grundkante der Länge $c$ kippt. Bei dem Kippen verliert der Quader an Energie, so dass $E_{kin}'=k\cdot E_{kin}$ gilt. Man kann annehmen, dass der Quader sich, sobald $\alpha=0$ dieser sich auf der anderen Kante befindet. Das Drehmoment um die erste Kante an dieser Kante wird vollständig in ein Drehmoment um die andere Kante umgewandelt.
\end{abcenum}
\textit{Aufgabe sollte nocheinmal korrigiert werden und mit Bild versehen. Es ist eine Gedächtnisabschrift}
\end{problem}

\ifx \envfinal \undefined
\end{document}
\fi
 